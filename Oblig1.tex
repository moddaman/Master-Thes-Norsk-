
\documentclass[phd,tocprelim]{cornell}

\let\ifpdf\relax
\usepackage{url}
\usepackage{graphicx}
\usepackage{color}
\usepackage{float}

\usepackage[section]{placeins}
\usepackage{graphicx,pstricks}
\usepackage{graphics}

\usepackage{moreverb}
\usepackage{subfigure}
\usepackage{epsfig}
\usepackage{txfonts}
\usepackage{multirow}
\usepackage{todonotes}
\usepackage{glossaries}
\usepackage[none]{hyphenat}
\usepackage{setspace}
\usepackage{hyperref}
\usepackage{listings}
\usepackage{minted}
\usepackage{pdfpages}



\graphicspath{ {Images/} }

\usepackage[utf8]{inputenc}

\tolerance=9999


\bibliographystyle{plain}
%\bibliographystyle{IEEEbib}

\usepackage{listings}
\usepackage{color}

\makeglossaries

\definecolor{lightgrey}{rgb}{236,236,236}
\definecolor{dkgreen}{rgb}{0,0.6,0}
\definecolor{gray}{rgb}{0.5,0.5,0.5}
\definecolor{mauve}{rgb}{0.58,0,0.82}
\renewcommand{\topfraction}{0.85}
\renewcommand{\textfraction}{0.1}
\renewcommand{\floatpagefraction}{0.75}
\renewcommand{\chaptername}{Kapittel}
\renewcommand{\abstractname}{Sammendrag}
\renewcommand{\acknowledgements}{Takk}
\renewcommand*\listfigurename{Figurliste}
\renewcommand\listoflistingscaption{Kodeliste}
\renewcommand{\bibname}{Referanser}
\renewcommand{\figurename}{Figur}

\title{Forbedret brukervennlighet innen alternativ og supplerende kommunikasjon for mennesker med forsinket 
eller avvikende språk- og kommunikasjonsutvikling} 

\author{Morten Holst Øvrebø / }
\pagenumbering{roman}

\begin{document}


\begin{figure}[ht!]
\centering
\includegraphics[width=80mm]{HIB_sort_hovedlogo_engelsk}
\end{figure}


\maketitle
\frontmatter
%\section*{Forord}


\begin{abstract}

Denne rapporten tar for seg prosessen rundt utviklingen av en high-fidelity prototype basert på en eksisterende programvare og testing av nye funksjoner for økt brukervennlighet. 

Tobii Dynavox har en programvare som heter Sono Flex som skal hjelpe unge mennesker som helt eller delvis mangler tale å kommunisere ved hjelp av øyesporing. De ønsket å utforske to ting.  Muligheten for å utvikle programvaren på en mer moderne plattform. Implementere og teste hvilke påvirkning animasjoner og lyd har på målgruppen.

Programvaren ble bygget på en ny plattform med hovedfunksjonaliteten tilgjengelig. Koden har følgt en såpass god standard at det skal være mulig å videreutvikle den til et en fullstendig programvare. I tilegg til hovedfunksjonaliteten ble det også implementert flere animasjoner. 

Til slutt ble det også utført en test på programvaren. Denne ga ikke nok data til å konkludere hvorvidt animasjoner og lyd ga økt brukevennlighet, men nok til at en person kan bruke den til kommunikasjon.



1
\end{abstract}


\begin{Acknowledgements}


Tusen takk til Tobii Dynavox som har hjulpet med å fullføre denne rapporten. Med en spesiell takk til Morten Mjelde fra Tobii, for gode ideer og tilbakemeldinger gjennom oppgavetiden. 

En stor takk til mine veiledere Harald Soleim, Atle Geitung, Remy Monsen og Jon Eivind Vatne for veiledning jeg ikke kunne ha greid meg uten.

For hjelp til å rekruttere testere vil jeg takke Solveig Kalgraf og Marie Brandvoll Haukenes ved HiB og Anne Moen ved pedagogisk-psykologiske tjenesten i Arna. For gjennomføring og tillatelse til å teste på Krohnengen barneskole ønsker jeg å rette en stor takk til Monica Storvik.

Til slutt ønsker jeg å vise takknemlighet til barna som deltok i testen og til deres foreldre for tillatelse.

\end{Acknowledgements}




\newglossaryentry{isaac}{name=ISAAC, description={ International Society for Augmentative and Alternative Communication}}

 \newglossaryentry{aske}{name=ASK, description={Alternativ og Supplerende kommunikasjon}}
 
  \newglossaryentry{PCCR}{name=PCCR, description={Pupil Centre Cornea Reflection, er en teknikk som brukes til ikke-forstyrrende øyesporing}}

  \newglossaryentry{WPF}{name=WPF, description={Windows  Presentation  Foundation}}
  \newglossaryentry{XAML}{name=XAML, description={Extensible Application Markup Language}}
 
\printglossary[title=Ordliste, toctitle=Beskrivelse av ord og forkortelser]
\contentspage
\figurelistpage
\listoflistings

\normalspacing \setcounter{page}{1} \pagenumbering{arabic}
\pagestyle{cornell} \addtolength{\parskip}{0.5\baselineskip}

\mainmatter


\chapter{Introduksjon}


\section{Bakgrunn}
\label{sec:motivasjon}

Tale og språk tillater oss å rekke ut til andre og leve tilfredsstillende liv som uavhengige medlemmer av samfunnet. Språk gir oss identitet, felleskap og tilhørighet. Det er derimot flere som blir hindret i å uttrykke seg gjennom tradisjonelle kommunikasjonsformer på grunn av ulike funksjonshemninger \cite{tobii}. De har derfor et behov for alternativer, for å kunne kommunisere. Vi kaller denne formen for kommunikasjon Alternativ og Supplerende Kommunikasjon \gls{aske}.

Folk med syn eller hørselsskader har tatt i bruk gester, tegnstøtte eller tegnspråk. Andre har måttet bruke mer håndgripelige hjelpemidler. Ett kjennetegn ved disse formene for kommunikasjon er at de krever at brukeren har muskelkraft. Dette kravet utelukker  personer med ALS, cerebral parese (CP), autisme og afasi eller de som har hatt hjerneslag. Folk med disse funksjonshemningene har ofte motoriske utfordringer som hindrer dem fra nettopp verbal og kroppsspråklig kommunikasjon. 

Ved hjelp av data- og øyestyrings-teknologi er det mulig for flere av disse menneskene å kommunisere. Feltet er relativt ungt og nåværende forskning har hovedsakelig blitt gjort på voksne uten funksjonshemninger \cite{aac}. I dette prosjektet vil fokus bli rettet mot barn som ikke har tilgang på tradisjonelle kommunikasjonsformer. 


\section{Motivasjon}
\label{sec:goal}

I dette prosjektet skal vi lage et digitalt system som hjelper barn med komplekse kommunikasjonsbehov å kommunisere. Systemet som skal utvikles tar utgangspunkt i et eksisterende program som heter Sono Flex (se figur ~\ref{fig:SonoFlex}). Programmets hovedfunksjon er å konvertere tekst og symboler til tale, men inneholder også et rikholdig utvalg av funksjoner for læring, omgivelseskontroll og elektronisk fjernkommunikasjon \cite{TobiiCommunicator}. Formålet med systemet er å hjelpe brukeren å kommunisere med symboler. Der et symbol representerer et ord eller et konsept som "hjem" eller "min mor". 

Sono Flex fungerer svært bra, problemstillingen er at når brukerens aktive vokabular vokser, så vil også antallet symboler øke. Dette gjør at det oppstår et behov for å dele symbolene inn i kategorier og flere visninger. Et barn vil da måtte navigere gjennom flere sider for å kunne skrive en ønsket setning. Det kan være belastende for et barn.

Light og Drager \cite{aac} argumenterer for at videre forskning innen ASK teknologi må fokusere på forbedret design, for bedre å kunne møte behovet fra unge barn og eldre nybegynnere. I dette prosjektet er målet å designe og prototype mulige løsninger som reduserer den mentale belastningen på brukeren, mens han bruker et stort vokabular, med eller uten en øyestyringsenhet. 

\begin{figure}[ht!]
\centering
\includegraphics[width=150mm]{SonoFlex2}
\caption{Skjermdump av ASK programvaren Sono Flex}
\label{fig:SonoFlex}
\end{figure}


\section{Målsetninger}

Utgangspunktet til oppgaven var å se på måter som kunne forbedre designet til Sono flex. Det ble derimot tidlig i startfasen avklart at Tobii ikke ønsket at utviklingen skulle skje på deres eksisterende kode. Grunnen er at den eksisterende teknologien som programvaren er bygget på begynner å bli utdatert. Den eksisterende teknologien og koden er i god stand, men ikke optimal for de ønskede forbedringene. 

Målet med oppgaven har derfor blitt todelt. Det vil si at i tillegg til å bygge ut funksjoner og teste disse på den eksisterende programvaren, vil også programvaren som disse skal fungere på utvikles. Første delmål av oppgaven vil derfor være å utvikle programvaren. Andre delmål vil være å teste ulike design og funksjoner.

\subsection{Første delmål: Prototype}

På grunn av begrenset med tid og ressurser ville det ikke være mulig  utvikle en fullverdig versjon av den eksisterende løsningen Sono Flex. Planen ble derfor å bygge en high-fidelity prototype hvor kun hovedfunksjonaliteten var prioritert. High-fidelity vil si at prototypen er svært likt det endelige produktet, med mye detaljer og funksjonalitet \cite{Usabi1:online}. En high-fidelity prototype er såpass lik sluttproduktet at brukervennlighetstesting av den, vil gi sterke konklusjoner om hvordan atferd i det endelig produktet vil være. Med "reverse engineering" menes det at prototypen vil bli bygget ved å se og analyse funksjonaliteten til Sono Flex og ikke ved å se på koden. Det vil si at funksjonaliteten og det meste av utseende ville være det samme, men at implementasjonen og teknologien ville bli annerledes.


Når en først skulle starte et nytt prosjekt ble det bestemt at det burde fokuseres på kodekvalitet. God kode hadde uansett vært i fokus, men i denne sammenheng menes det litt utover det. Istedenfor å kun ha hovedfokus på å implementere funksjonalitet, så bør den være implementert slik at den følger standarder og mønster. Tankegangen var at ved å gjøre dette ville det være mulig å bruke prototypen i fremtidige prosjekter. Det økte også muligheten for at prototypen kunne videreutvikles til et ferdig prosjekt. 

Minstekravet til prototypen var at den skulle gi brukeren mulighet til å sette sammen enkle setninger å spille disse av som lyd. De ulike ordene skulle bli delt inn i tilhørende kategori. 


\subsection{Andre delmål: Forbedret brukervennlighet}
\label{sec:ResearchQuestion}

For å utforske mulige løsninger som kan øke brukervennligheten til Sono Flex vil en rekke funksjoner og design implementeres og testes. Først skal vi se på hvordan animasjon og lyd kan hjelpe en bruker med å finne ønsket symbol raskere. Deretter ønsket vi å finne en måte å gjøre det mulig for en bruker å endre innstillinger kun ved å bruke øyene. Vi velger å kalle dette øyestyrt brukertilpasning. Til slutt vil vi utforske veier en kan organisere og kategorisere ord på, for best å legge tilrette for et barn. Listen nedenfor gir en  beskrivelse av det vi ønsker å utforske.

\begin{enumerate} 
\label{lst:features}
\item Animasjon. Undersøk mulige fordeler forskjellige visualiseringseffekter kan ha på navigering innen og mellom sider og kategorier.
\item Lydeffekter. Undersøk mulige fordeler forskjellige lydeffekter kan ha på navigering innen og mellom sider og kategorier.
\item Optimal organisering og kategorisering. Hvordan skal knapper bli organisert og kategorisert for best å legge til rette for og forenkle navigasjon
\item Brukertilpasning. Mennesker har forskjellige preferanser, derfor er det i de fleste programvarer mulig for en bruker å tilpasse etter ønske. I dette tilfelle ønsker vi å gjøre dette mulig gjennom øyestyring. Spesifikk vil vi se på mulighetene for tilpasning av symbolstørrelse, animasjonsfart, farger og generelle innstillinger. 
\end{enumerate}

Forskningen vil bli gjort ved å implementere et program basert på en eksisterende løsning. De ulike forbedringene beskrevet i listen vil bli integrert.

\section{Testing}

Som nevnt i seksjon \ref{sec:ResearchQuestion}, er oppgaven å forbedre en type kommunikasjonsprogramvare for mennesker med komplekse kommunikasjonsvansker. For å undersøke virkningene til de forskjellige funksjonene, vil de prøves ut på en testgruppe.

For å få til dette, vil et utvalg av deltakerne prøve systemet uten tilleggsfunksjonene, mens de gjenværende vil forsøke med funksjonene. Mens deltagerne kjører programmet vil deres interaksjon med programmet bli lagret i en logg. Informasjonen fra undersøkelsen vil vise hvordan brukeren navigerte og hvor lang tid de brukte. Dette kan igjen brukes til å verifisere om en funksjon er en forbedring eller ikke.

\subsection{Målgruppe}

I kapittel \ref{sec:motivasjon} nevnes det at det er flere unge mennesker som helt eller delvis mangler tale. Som en konsekvens av dette har de behov for andre uttrykksformer for å kommunisere. Alternative uttrykksformer kan være håndtegn, symboler eller fotografier. Personer bruker ASK enten fordi det er et behov for å erstatte talen eller for å supplere utydelig eller svak tale.

International Society for Augmentative and Alternative Communication (\gls{isaac}) \cite{HvaErASK} definerer ASK som alt som hjelper en person til å kommunisere effektivt når tradisjonelle måter å kommunisere på ikke strekker til.

Målgruppen er personer som har behov for ASK systemer. Mer spesifikk vil programvaren være rettet mot: (A) Mennesker som ikke har mulighet for tale- og kroppsspråk,  (B) personer som ikke håndterer skriftspråk og (C) de som er nybegynnere på symbolbasert kommunikasjon. Dette gjelder hovedsakelig barn i alderen 2 til 5 år, men også eldre med mentale begrensninger. Videre i dette prosjektet vil personer som møter disse kriteriene bli omtalt som barn med komplekse kommunikasjonsbehov. 


\section{Oppsummering}

Tobii Dynavox vil utforske potensielle forbedringer i deres programvare. Problemet er at den eksisterende teknologien ikke er tilstrekkelig. Dette gjør at før vi kan utvikle og teste mulige forbedringer, må en prototype utvikles. Målet for oppgaven blir derfor først å utvikle prototype, deretter implementere mulige forbedringer.

\chapter{Bakgrunn}
I dette kapittelet vil vi gå dypere inn i hva vi ønsker å finne ut og hvordan vi skal prøve å finne dette ut. 

\section{Tidligere Forskning}


\subsection{Animasjoner}


Animasjoner i programvare brukes som et virkemiddel for å tiltrekke seg oppmerksomhet, underholde eller demonstrere noe. Det kan enten være en kort animasjon der en knapp tiltrer til en side når man svever over den med musen eller det kan være av den lengre sorten som en flash-video.  

Animasjon er i hovedsak alt som beveger seg i applikasjonen. Selv om dynamikken som animasjoner kan fungere som et positivt virkemiddel viser undersøkelser gjort av Nielsen og Loranger \cite{NielsenBok}  at for mye blinkende og bevegende elementer kan slite ut brukeren og gjør det vanskeligere å fokusere på oppgaven. Med animasjonene som er implementer ønsker vi å undersøke dette. Hvilke animasjoner fungerer som et hjelpemiddel og med det gir verdi til applikasjonen og hvilke som er distraherende for brukeren og gir en uønsket effekt. 
 


\subsection{Organisering}

I 1926 kartla Smith\cite{Smith} barns vokabularutvikling fra alderen 1 til 5 år. Resultatene (se tabell \ref{fig:BarnVak}) viste at allerede i en alder av 3 er det vanlig å ha en gjennomsnittlig forståelse av cirka 900 ord. Forskningen er 88 år gammel, men viser omfanget av utfordringen barn har i møte med symbolbasert kommunikasjon. Mens et barn med taleevne kun trenger å finne ordet i hukommelsen, må barn som bruker symboler først finne det og deretter lokalisere det representative symbolet. 

I applikasjonen som skal utvikles vil symbolene bli presentert i en tabell. Antall symboler som får plass i tabellen begrenses av skjermens fysiske størrelse og at de må være store nok til at brukeren har mulighet til å se  og samhandle med dem. En slik tabell med symboler vil herfra bli referert til som en side. 

For hver setning brukeren vil uttrykke må han navigere seg igjennom flere av disse sidene for å finne symbolene som representerer ordene i setningen. Det er derfor essensielt at organiseringen legger til rette for akkurat dette, og ikke vanskeliggjør barnets evne til å lokalisere, velge og bruke symbolene.

\begin{table}[h]
\begin{tabular}{llllllllll}
\hline
Alder (År, Måneder) & 1 & 1,6 & 1,9 & 2,0 & 2,6 & 3,0 & 3,6  & 4,0  & 5,0  \\ 
Antall ord          & 3 & 22  & 118 & 272 & 446 & 896 & 1222 & 1540 & 2072 \\ 
Økning              & 2 & 19  & 96  & 154 & 174 & 450 & 326  & 330  & 532  \\ \hline
\end{tabular}
\caption{Tabell som viser vokabular vekst hos barn.  Smith \cite{Smith} sitert av Dale \cite{Dale} }
\label{fig:BarnVak}
\end{table}


Drager og Light \cite{aac} viser til at det inntil nylig var gjort lite forskning på layouts og organisering for barn eller om hvilke faktorer som spiller inn når det gjelder lokalisering og bruk av målobjektene. Målobjektet er symbolet som brukeren ønsker å uttrykke. Wilkinson og Jageroo \cite{Wilkinson2006} har undersøkt hvilke påvirkning farge har som en faktor når det kommer til organisering og hvordan elementer skal fordeles. De kom frem til at farge spiller en viktig rolle innen visuell prosessering  og hukommelse. Ved å legge til en farge ved elementene i tabellen over symboler, påvirket det nøyaktigheten og effektiviteten til barn i 4-5 års alderen i å finne målobjekt. Eksempelvis kan man gi symboler som representerer verb en grønn farge, adjektiv en blå, pronomen gul o.s.v. Fenomenet kalles Fitzergald Key og gjør at barn mer presist og raskere lokalisere målobjektet.

Scally \cite{Scally} argumenterer for at farge ikke nødvendigvis er den eneste variabelen som må vurderes. Andre skjermvariabler som kan påvirke læring og bruk er: bakgrunn, kanter/grenser, form, tekstur, størrelse, posisjon, bevegelse og animasjon. Videre undersøkelser må til for å avgrense effektene av disse funksjonene for å kunne optimere designet til ASK-systemer.


\subsection{Navigasjon}
\label{subsec:navigasjon}

Hvis en tar med alle typene frukt som finnes vil det ikke være plass til alle symbolene på en side. Dette gjør at de må deles opp over flere sider. Konsekvensen er at brukeren må ha mulighet til å navigere mellom disse sidene for å finne ønsket symbol. Siden antallet symboler det er plass til på hver side begrenses av skjermstørrelsen og brukerens syn, er spørsmålet hvor mange symboler bør plasseres på hver side med tanke på effektivitet. 
Drager og Light \cite{aac} undersøkte nettopp dette. De kom frem til at det er vanskeligere for et barn å lokalisere korrekt side fra en meny med 4 valg, enn å finne korrekt symbol fra korrekt side med 12 - 30 valg.


For barn kan det å navigere være ekstra vanskelig. Dette kommer av flere grunner: (a) de må ha en konseptuell modell av de gjemte sidene i systemet i minnet. Med andre ord forstå hvordan de mest effektivt kan komme seg fra forsiden til ønsket symbol. (b) De må forstå forholdet mellom representasjon brukt på menysiden og de gjemte sidene i vokabularet. Er det symbolet med bilde av et kjøkken eller av en butikk som leder til symbolet eple? I denne rapporten vil disse utfordringene bli undersøkt.





\chapter{Tobii Sono Flex}
I dette kapittelet vil den eksisterende løsningen sono flex bli diskutert. Det vil bli gitt en grundig analyse av problemstillingen og fremgangsmåte presentert.

\section{Tobii Sono Flex}
\label{chap:Tobii-Sono-Flex}

Programvaren som det tas utgangspunkt i heter Tobii Sono Flex,  og er et systematisert symbolforråd og et verktøy for alternativ og supplerende symbolspråk. Sono Flex har som mål å tilby et språk til personer som ikke enda kan lese og skrive. 

Applikasjonen fungerer som et tastatur, men istedenfor bokstaver er knappene ord med en visuell representasjon av ordet. Brukeren trykker på knappene som utgjør setningen han vil uttrykke, så vil programvaren gjøre om setningen til tydelig tale.  Systemet er spesielt utviklet for barn og unge med med sammensatte kommunikasjonsvansker, som trenger et ordforråd for å videreutvikle språk- og kommunikasjonsferdigheter. Sono flex kan ses på som en nybegynnerpakke med en lav læringskurve som skal gjøre brukeren klar for mer avanserte systemer. 


\subsection{Brukergrensesnitt}

Brukergrensesnittet til Sono Flex består av to hovedkomponenter: en menylinje og en symboltabell. 


\subsubsection{Menylinje}

Figur \ref{fig:menylinje} viser menylinjen.  Denne består av 5 elementer.  Kun de midterste er av interesse, disse  er statiske og følger applikasjonen hele tiden. Det hvite feltet i midten viser symbolene som brukerene har trykket på, og vil herved bli referert til som setningslisten. Symbolene som vises vil komme ut i form av tydelig tale ved å trykke på selve feltet. Knappen på venstre side av setningslisten (clear all), vil ved interaksjon tømme setningslisten. Mens knappen på høyre side vil kun fjerne det siste symbolet brukeren trykket på.


\begin{figure}[ht!]
\centering
\includegraphics[width=100mm]{menylinje}
\caption{Skjermdump av menylinjen til programvaren Sono Flex}
\label{fig:menylinje}
\end{figure}


\subsubsection{symboltabell}
\label{subsubsec:symboltabell}

Figur \ref{fig:symbolgrid} viser applikasjonens symboltabell. Dette komponentet består av en tabell på 7 kolonner og 4 rader,  noe som gir 28 celler. I hver celle er det et Knapp bestående av et symbol og en tekstlig beskrivelse av symbolet. Ved å trykke på knappen vil en av tre ting skje avhengig av hvilken type knappen er av. Hvis knappen representerer et ord( "jeg",  "løpe",  "kake" o.s.v. ) så vil symbolet og medfølgende tekst vises i setningslisten i menylinje.  Hvis knappen har underliggende knapper som eksempelvis en ordklasse(verb,  substantiv)  eller kategori ("mat",  "frukt") så vil applikasjonen bytte ut de eksisterende symbolene i tabellen med ordene i ordklassen. Den siste typen symbol, er et navigasjonssymbol. Denne forekommer kun hvis det er for mange symboler i forhold til hvor mange det er plass til. Eksempelvis er det plass 28 symboler i tabellen, hvis da en kategori inneholder mer enn dette vil disse måtte fordeles over flere sider. Navigasjonssymbolet vil da være representert for å kunne bla mellom de ulike sidene.


\begin{figure}[ht!]
\centering
\includegraphics[width=100mm]{symbolgrid}
\caption{Skjermdump av symboltabellen til programvaren Sono Flex}
\label{fig:symbolgrid}
\end{figure}


\section{Kommunikasjonsform: Symboler}

De som ikke har mulighet til å bruke skrift som kommunikasjonsform kan bruke tegnsystemer. Det eksisterer tre typer tegnsystemer: håndtegn(manuelle tegn) som innebærer å bruke håndbevegelser, materielle tegn vil si at en bruker fysiske objekter som brikker eller figurer og den siste er; grafiske tegn som innebærer at en bruker symboler. I denne rapporten er vil symboler bli brukt som kommunikasjonsform. Her representerer symboler et ord, frase, uttrykk eller setning. Ifølge ISAAC \cite{Tegnsystemer} er grafiske tegn brukt av mennesker med store bevegelsesvansker som gjør at de har utfordringer med å lage manuelle tegn, og mennesker med forståelsesvansker som følge av lærehemning.

De vanligste grafiske tegnsystemene på markedet er fotografi, pictogram, Picture Communication Symbols(PCS), Widgit, SymbolStix og Bliss \cite{GrafiskTegn}. I prosjektet brukes SymbolStix (se figur \ref{fig:katt}). Grunnen til dette er at Sono Flex bruker dette tegnsystemet.


\begin{figure}[ht!]
\centering
\includegraphics[width=50mm]{katt}
\caption{Et eksempel på et symbol som blir brukt i det grafisk tegnsystemet SymbolStix}
\label{fig:katt}
\end{figure}

\section{Interaksjonsform: Øyestyring}

Menneske-maskin-interaksjon fungerer ved at en person gir maskinen en kommando, og maskinen svarer med en respons. For at et menneske skal kunne gi en maskin ordre, er det nødvendig at maskinen har en inputenhet som tolker beskjedene fra brukeren, slik at at datamaskinen forstår dem.  Som regel er dette tradisjonelle enheter som tastatur og mus. Det eksisterer derimot mangfoldige måter å samhandle med en maskin på. I denne rapporten vil brukerinput bli gitt ved en øyestyringsenhet.  Ved å fange opp brukerens skuepunkt av et videokamera og infrarøde lys, er det mulig for maskinen å beregne hvor på monitoren personen ser. Dette gjør interaksjon mellom dem mulig. En knapp vil for eksempel bli aktivisert ved at brukeren ser på den en hvis periode. 

\subsection{Brukerinteraksjon}

Sono Flex tilbyr to måter for brukerinteraksjon, mus og øyestyring. Mus fungerer som vanlig ved at brukeren svever med musepekeren over ønsket knapp og venstreklikker for å aktivere. Ved øyestyring må brukeren fokusere blikket på ønsket knapp en gitt tid for at applikasjonen skal tolke det som et klikk. 

Det som skjer er at med engang brukeren fokuserer blikket på en knapp så starter en nedtelling. Figur \ref{fig:knapp-interaksjon} viser hvordan brukeren presenteres for hvor mye av nedtellingen som gjenstår.  Hvis brukeren ikke flytter blikket før nedtellingen har nådd null tolkes dette som et klikk.
For å gi brukeren beskjed om et godkjent klikk så dannes det en rød firkant rundt knappen. 

\begin{figure}[ht!]
\centering
\includegraphics[width=50mm]{Knapp-interaksjon}
\caption{Skjermdump som viser hvordan nedtellingen på en knapp ser ut. Når den røde sirkelen er komplett oppfattes det som et klikk}
\label{fig:knapp-interaksjon}
\end{figure}


\section{Organisasjon og Navigasjon}

Som nevnt i seksjon \ref{subsec:navigasjon} så vil et barns vokabular være så stort at en nødvendigvis må fordele ordene over flere sider. Sono Flex har eksempelvis 106 ord under kategorien "Ting". Med tanke på at det maksimalt er plass til 28 ord på hver side så må disse fordeles og det må finnes en måte å navigere sidene. Løsningen har blitt en svært flat struktur med et hierarki på maksimalt 2 nivåer, ergo det finnes ikke underkategorier. Figur \ref{fig:hieraki-ting} viser hvordan dette fungerer i praksis. Når man trykker på kategorien "ting" så fylles tabellen med 27 ord som passer inn i kategorien "ting". Hvis man ikke finner ønsket ord på den første siden, navigerer man videre med knappen "neste side" og ordene i tabellen erstattes av nye ord. 


\begin{figure}[ht!]
\centering
\includegraphics[width=140mm]{Symbgrid}
\caption{Skjermdump hierakiet til Sono Flex}
\label{fig:hieraki-ting}
\end{figure}


\chapter{Øyesporing}

I dette kapittelet blir det gitt enkel forklaring over øyet og hvordan der er mulig å spore det ved hjelp av en øyesporingsenhet. Til slutt vil enheten som brukes i prototype og hvordan den brukes bli presentert.


\section{Hvordan fungerer øyet?}

Å forklare hvordan øyet fungerer i detalj er utenfor denne rapportens omfang. Det vil derimot gitt en enkel forklaring av hvordan det fungerer og de mest nødvendige konseptene.

\subsection{Synsfelt}

I en artikkel skrevet av Tobii \cite{Calibration} sammenlignes øyet med et fotoapparat på grunn av dens mange likhetstrekk. Det hele starter ved at når lys treffer et objekt så reflekteres det. Og på samme måte som et kamera så fanges det reflekterte lyset opp av en linse, som igjen prosjekterer det på en lyssensitiv overflate. Men i motsetning til et kamera er ikke denne overflaten like sensitiv overalt i øyet. Dette gjør at menneske kan tilpasse synet etter hvor mye lys som er tilgjengelig. En bieffekt er at menneske kun kan se klart i begrensede områder av synsfeltet. Dette er illustrert i Figur \ref{fig:visueltArea}, som viser hvordan synsfeltet hos menneske er delt inn etter klarhet. Det innerste området notert ved bokstaven F representerer det foveale området, også kjent som skarpsynet. Dette er den delen av synsfeltet man fokuserer på og som man derfor oppfatter som klarest. Det er hovedsaklig fra dette området visuell data hentes fra. Området deklarert med bokstavene Pf i figuren viser det parafovela området. Som er et overgangsområde og kjennetegnes ved at uskarpheten gradvis øker til man kommer til det perifere området vist som P i figuren. Det perifere området, også kjent som sidesynet, er det mest uskarpe området og fungerer kun bra til å fange opp bevegelser og kontraster.

\begin{figure}[ht!]
\centering
\includegraphics[width=65mm]{fovealArea}
\caption{Bilde/Illustrasjon av menneskelige synsfelt \cite{VisualImage}}
\label{fig:visueltArea}
\end{figure}

\subsection{Øyebevegelser}

Det foveale området er som tidligere nevnt det området det registreres mest visuell data. Ulempen er at området kun står for rundt 2 grader av synsfeltet \cite{Backg0:online}. Så hvis en ønsker å hente detaljert informasjon fra andre deler av synsfeltet, må det foveale området flyttes ved å bevege øyene. De ulike øyebevegelsene som brukes er beskrevet i Listen under er hentet fra nettsiden synstap \cite{sanse7:online}.

\subsubsection{Grovmotoriske}
\begin{itemize}
\item Akkomodasjon – øyets evne til å se klart på forskjellige avstander, det å kunne se skarpt når en flytter blikket hurtig fra avstand til nært og omvendt
\item Følgebevegelser – kunne følge en gjenstand med blikket i alle retninger uten å bevege hodet
\item Konvergens – kunne holde blikket samlet på nært hold
\item Stereosyn – evnen til å se avstand og dybde
\end{itemize}
\subsubsection{Finmotoriske}
\begin{itemize}
\item Sakkader – små forflytninger med blikket ved for eksempel lesing
\item Minisakkader – minibevegelser av øyet (dirring), som må være tilstede for at det skal sendes informasjon til hjernen
\item Antisakkader – evnen til å undertrykke en øyebevegelse
\item Fiksering – evnen til å holde blikket stødig festet på ett punkt
\end{itemize}



\section{Hva er øyesporing?}

Øyesporing er prosessen med å måle hvor en person ser, eller bevegelsen til et øye i forhold til hodet \cite{Eye t4:online}. Ved å gjøre dette kan man finne ut hvilke elementer brukeren ser på, hvor lenge han ser på det og hvordan han beveger blikket. 


\subsection{Bruksområder}

Øyesporing kan anvendes på flere måter. Blant annet så brukte Alfred L. Yarbus øyesporing det til forskning. I sin artikkel fra 1967 \cite{wexle4:online} beskriver han hvordan et subjekts øyebevegelser blir påvirket av oppgaven han skal gjennomføre. Bilde \ref{fig:yarbus} viser hvor forskjellig en person ser på et bilde basert på hvilken oppgave han er tildelt. I tilegg til vitenskap brukes øyesporing også til markedsundersøkelser, reklame, brukervennlighets-testing og mer \cite{Case2:online}. Det som kjennetegner de nevnte bruksområdene er at brukeren ikke aktivt foretar seg noe, han bruker ikke øyene til å gjøre noe annet enn å se. Noe som er annerledes fra anvendelsen i denne oppgaven, hvor øyene i tilegg til å se, brukes til interaksjon.



\begin{figure}[ht!]
\centering
\includegraphics[width=100mm]{Yarbus_The_Visitor}
\caption{Bilde som viser hvordan et subjekts øyebevegelser påvirkes av oppgaven \cite{Yarbu2:online}}
\label{fig:yarbus}
\end{figure}


\subsection{Pupil Centre Cornea Reflection}

Øyesporing er som tidligere nevnt teknikken brukt til å fange og måle øyebevegelser. Det finnes derimot flere fremgangsmåter. I denne oppgaven vil det brukes en ikke-forstyrrende øyesporingsenhet. Dette gjør at brukeren i prinsippet ikke skal legge merke til enheten. For denne typen øyesporing er det mest vanlig å bruke en teknikk som heter Pupil Centre Cornea Reflection (\gls{PCCR}) \cite{Calibration}. Teknikken fungerer ved at en lyskilde belyser øyet for at refleksjonene skal bli klare og synlige. Et kamera tar deretter bilde av refleksjonene fra øyet. Bildet blir så brukt til å identifisere lysets refleksjon på hornhinnen og pupillen. Når en vet vinkelen mellom hornhinnen og pupillen er det mulig å regne ut en vektor. Vektoren sammen med andre geometriske egenskaper ved refleksjonene gjør det mulig å kalkulere ut blikkretningen(Der brukeren ser) \cite{Calibration}.


\section{Tobii PCEye Go}

I prototypen har vi valgt å ta i bruk øyesporingsenheten Tobii PCeye GO \cite{Contr4:online}, som er vist på bilde \ref{fig:tobiiPc}. Enheten kommer separat og kobles til datamaskinen via USB. Denne enheten ble valgt fordi den brukes i Sono Flex og fordi kontaktpersonen i Tobii allerede hadde god erfaring med enheten. Den har også den foredelen at den er av den ikke-forstyrrende typen. Det vil si at en bruker i teorien ikke vil plages av enheten. Noe som hadde vært tilfelle ved å bruke brillene vist på bilde \ref{fig:tobii_glasses} til øyesporing.


\begin{figure}[ht!]
\centering
\includegraphics[width=50mm]{TobiiEyeGo}
\caption{Bilde av øyesporingsenheten Tobii PCEye Go}
\label{fig:tobiiPc}
\end{figure}

\begin{figure}[ht!]
\centering
\includegraphics[width=50mm]{Tobii_Glasses}
\caption{Øyesporingsenhet i form av briller som brukeren har på seg}
\label{fig:tobii_glasses}
\end{figure}


\subsection{Tobii Eye Control API }

Figur \ref{fig:overview} viser hvordan blikk interaksjonsserveren eksponerer funksjonalitet til en klient-applikasjon gjennom APIet kalt Tec API. For å ta i bruk TecAPIet tilbys to aksess punkter. Et gjennom .NET plattformen kalt TecClient og et for C dynamisk link library kalt MPACI.  Den praktiske betydningen, er at man kun kan bruke APIet ved å skrive i C eller .NET teknologier. I denne rapporten vil kun sistnevnte være interessant, altså .NET APIet kalt TecClient.


\begin{figure}[ht!]
\centering
\includegraphics[width=100mm]{SoftwareArchitectureOverview}
\caption{Bilde som viser programvare arkitekturen til blikk programvaren}
\label{fig:overview}
\end{figure}


\subsection{TecClient}

For å gi tilgang til API funksjoner er TecClient komponentet for .NET delt inn i verktøys klasses(toolbox classes). Figur \ref{fig:toolbox} viser de ulike verktøyene som er tilgjengelig. 

\begin{figure}[ht!]
\centering
\includegraphics[width=100mm]{Toolbox}
\caption{Verktøy klassser som er tilgjengelige gjennom TecClient komponenten verktøyskasse}
\label{fig:toolbox}
\end{figure}


\textbf{EyeTracker} Gir informasjon om den aktuelle øyesporingsenheten. 

\textbf{Kalibrering} 
For optimal øyesporing med Tobii PCeye Go må enheten kalibreres for hver bruker. Dette verktøyet går gjennom en prosedyre for å måle karakteristikk ved personens øyer som brukes igjen til å lage en fysiologisk 3d modell for å kalkulere hvor brukeren ser \cite{www.t5:online}. Bilde \ref{fig:kalibre} viser et bilde fra kalibreringsprosessen, hvor brukeren blir bedt om å se på de ulike punktene.

\begin{figure}[ht!]
\centering
\includegraphics[width=100mm]{kalibrering}
\caption{ \cite{VHye88:online}}
\label{fig:kalibre}
\end{figure}


\subsubsection{Settings}
Settings komponente gir tilgang til brukerprofiler og operasjoner som gjør at man kan skifte mellom dem. For eksempel er det naturlig at en bruker ikke vil gå gjennom kalibreringsprosessen hver gang han vil ta i bruk enheten. I dette komponentet kan karakteristikken fra kalibreringen lagres i brukerprofilen som er er koblet til brukeren, slik at han slipper dette. 


\subsubsection{Trackstatus}
Tilbyr metoder, egenskaper og hendelser for å kontrollere sporingsstatus vinduet. Bilde \ref{fig:track} viser hvordan dette komponentet kan brukes til å vise brukeren om enheten fanger opp øyene og hvordan de er iforhold.

\begin{figure}[ht!]
\centering
\includegraphics[width=100mm]{Trackstatus}
\caption{}
\label{fig:track}
\end{figure}

\subsubsection{Gaze}
Eksponerer rå blikkdata, både filtrert og ufiltrert. Det vil si at man kan kan finne ut hvor på skjermen en bruker ser i form av X og Y koordinater.


\subsubsection{Region / Interaction regions}

Ifølge dokumentasjonen er en interaksjon region er en geometrisk region på skjermen som en bruker kan interagere ved å se på den (dokumentasjon). En knapp i applikasjon vil typisk bli definert som en interaksjon region. Ved å definere en region kan man lytte til hendelser fra dem. De to viktigste hendelsene er \textbf{Focus} og \textbf{Activation}. Focus hendelsen vil aktiveres når en bruker skuer innenfor grensene til en region. En bruker vil bli opplyst om at han er innenfor en slik region ved hjelp av en indikator. Denne indikatoren vil typisk gi brukeren et signal om at en nedtelling har startet. Når nedtelling er ferdig vil Activation hendelsen bli avfyrt. Activation er konseptuelt det samme som å klikke med en datamus, og responsen vil normalt være den samme.





 
 
\chapter{Prototypen} 

I dette kapittelet vil det bli gjort rede for utviklingen av prototypen og valg som ble gjort under prosessen. 
 
 
\section{Utgangspunkt} 
\label{sec:utgangspunkt} 
 
Målet med forskningen er å undersøke potensielle forbedringer funksjonene beskrevet i seksjon \ref{sec:ResearchQuestion} kan ha på den eksisterende løsningen Tobii Sono Flex (\ref{chap:Tobii-Sono-Flex}). Det mest naturlige valget ville vært å implementert og endret nødvendig kode i den eksisterende kodebasen. Tobii Dynavox ønsket derimot å utforske andre muligheter. Blant annet fordi den eksisterende teknologien hadde flere begrensninger som ville gjort det vanskelig å fått implementert mye av funksjonalitet som skulle undersøkes. Spesielt gjelder dette animasjonene. Dette gjorde at utviklingen ble startet som et nytt prosjekt og at første del ble å lage en high-fidelity prototype ved å bruke "reverse engineering".


 
\section{Kravspesifikasjon} 
 
Å implementere all funksjonaliteten som Sono Flex har, ville vært for ressurskrevende. Det ble heller bestemt å fokusere på de mest nødvendige og heller gjøre dette skikkelig. Slik at det var mulig å bygge videre på koden i ettertid. Det mest nødvendig vil si hovedfunksjonaliteten, som er å la en bruker ha mulighet til å skrive setninger med symboler for så å gjøre om disse til naturlig tale. For å kunne gjennomføre dette er det flere mindre oppgaver som måtte implementeres. 

Under er kravspesifikasjonen. De ulike kravene er beskrevet etter prioritert, der de første er de mest nødvendige og kjent som kjernefunksjonalitet. Deretter vil funksjonalitet som hadde vært greit å ha, men som ikke er nødvendig for at applikasjonen skal kjøre, beskrevet.  
 
\subsection{Funksjonelle krav} 
 
 
\textbf{Øyesporing som interaksjon} - Det skal være mulig å bruke \underline{kun} øyene til å operere prototypen. En som tar i bruk øyesporing skal ha akkurat de samme mulighetene som han ville hatt ved å ta i bruk datamus. Effektiviteten skal være så lik som mulig mellom de to interaksjonsformene. 
 
\textbf{Logging} - I systemet skal det være mulig å kunne logge all interaksjonen en bruker gjør med programvaren. Denne funksjonen vil være nødvendig for å kunne gjennomføre testingen 
 
\textbf{Brukertilpasning} - Hver bruker skal ha mulighet til å kunne tilpasse programvaren etter sine preferanser. 
  
\textbf{Tale} - Systemet skal kunne gjøre om tekst til lyd. Det vil si at andre personer skal kunne forstå setningen brukeren har skrevet kun utifra lyden. Jo mer naturlig talen høres ut jo bedre.
  
\textbf{Norsk} - Systemet skal ha språkstøtte for norsk. Bør også legges opp til mulighet for å endre og legge til støtte for andre språk 
 
\textbf{Animasjon} - I prototypen skal det være animasjoner som blir aktivert når en bruker trykker på de ulike symbolene.  

\textbf{Lydeffekter} - Utenom talelyden, skal det også være lydeffekter som skal avspilles ved brukerinteraksjon.  
 
\textbf{Symbol} - For vært ord skal det være symbol som representerer ordet. En bruker skal i teorien, ikke ha behov for å lese ordet og skal kun i utifra ordet forstå hvilket ordet det representerer. 


 
\subsection{Ikke-funksjonelle krav} 
 
 
\textbf{Brukervennlighet} - Målgruppen består av barn med som har begrenset erfaring med å operere dataprogrammer. Det er derfor viktig at det legges vekt på det og programvaren utformes på en måte som er intuitiv for brukeren.   
 
 
\textbf{Fleksibilitet} - Kodebasen skal være tilrettelagt for vedlikehold og videreutvikling. Det er viktig at personer som ikke har deltatt i systemutvikling skal ha mulighet til å forstå koden og på den måten enkelt kan legge til og fikse funksjoner. Programvaren skal også legge oppp til at det er enkelt og legge til animasjoner, lyd og nye brikker. 
 
 \textbf{Responstid} -  Programvaren er kompleks og det tar lang tid å skrive med symboler, det er derfor viktig at systemet responderer kjapt og ikke gjør slik at oppgaven tar lenger tid. Slik at når brukeren trykker på noe skal det føles som om programvaren svarer momentant. 

\textbf{Personvern} - Brukerne av programvaren er en sårbar gruppe, det vil derfor være viktig at sensitiv informasjon om disse ikke kommer på avveie. Det skal i utgangspunktet ikke lagres sensitiv data, men hvis data lagres skal det lagres på en sikker måte. 
 
\section{Utvikling} 
 
I denne seksjonen vil teknologier og arbeidsområde bli beskrevet, som skal gi grunnlag for den neste seksjonen som vil gå mer innpå implementasjon valg og detaljer. 
 
 
\section{Programmeringsrammeverk} 

Programmeringsrammeverk er ikke nødvendig for å kunne bygge prototypen, men tilbyr flere gode fordeler. Blant annet lav-nivå kode som allerede har blitt bygget, testet og som har blitt brukt av andre programmerere, noe som øker påliteligheten og reduserer utviklingstiden \cite{Frame7:online}.

Til å utvikle prototypen ble det brukt .NET. Dette kom som et resultat av at
to faktorer. Den en var dokumentasjonen til øyesporingsenheten anbefaler det og alle eksemplene av bruk er i en .NET teknologi. Den andre var at rammeverket tilbyr flere gode brukergrensesnitt-teknologier som var ønskelig med tanke på at brorparten av utviklingen handler om det grafiske.


 
\subsection{.NET}
 
Arkitekturen til .NET er omfattende og som en kan se utfra figur \ref{fig:net-arkitektur} er det flere programmeringsspråk og komponenter en kan velge å ta i bruk. Rapporten vil derimot kun gi informasjon om de ulike komponentene fra .NET som ble brukt og er nødvendig for videre lesning. 
 
 
\begin{figure}[ht] 
\centering 
\includegraphics[width=140mm]{netframework45} 
\caption{Diagram som viser arkitekturen til .net rammeverket versjon 4.5} 
\label{fig:net-arkitektur} 
\end{figure} 
 
 
\section{Brukergrensesnitt-teknologi} 

Brukergrensesnitt-teknologi gir utvikleren tilgang på funksjoner og forhåndsdefinerte grafiske elementer som skal gjør utviklingen mer effektiv. Som en kan se utifra figur \ref{fig:net-arkitektur} og Microsoft's oversikt \cite{User1111:online} tilbyr .NET teknologier som Windows Store Apps, Windows Presentation Foundation(\gls{WPF}), Expression Blend 3 + SketchFlow, Windows Forms og Silverlight. Alle disse ble evaluert før et endelig valg ble foretatt:

\textbf{Windows Forms} ble utelukket ettersom dette er teknologien som Sono Flex er bygget på og som hadde begrensinger som gjorde at flere ønskede funksjoner ble vanskelig å implementere. 

\textbf{Windows Store Apps} hadde mye av den ønskede funksjonaliteten, samtidig som den fortsatt vedlikeholdes. Problemet var at for å distribuere slike applikasjoner brukes Windows Store. F å få publisert applikasjonen der, kreves godkjenning av Windows. En tidkrevende prosess. Dette i sammenheng med at disse applikasjonene ikke er kompatible med Windows 7\cite{Windo0:online} gjorde at denne teknologien ikke ble valgt. 

\textbf{Silverlight} er ifølge Microsoft sin blogg\cite{User1111:online} en kraftig utviklingsplattform for å lage "rike" media- og forretningsapplikasjoner for web, skrivebord og mobil. Fokus for teknologien er derimot på web og mobil \cite{Micro6:online}, og ettersom prototypen er en skrivebords applikasjon falt ikke valget på Silverlight.

Teknologien som til slutt ble valgt var \textbf{Windows Presentation Foundation}. WPF skal tilby utviklere en helhetlig programmeringsmodell for å bygge skrivebordsapplikasjoner for Windows som tar i brukergrensesnitt, media og dokumenter \cite{Windo777:online}. Grunnen til at teknologien ble valgt er:


\begin{itemize}
\item Fokus på skrivebord. I motsetning til blant annet Silverlight, er WPF designet og laget for å utvikle klient applikasjoner\cite{Windo777:online}.
\item Prioritert av Tobii PC Eye GO. I dokumentasjonen til øyesporingsenheten er det skrevet: "WPF  rammeverket er et svært bra verktøy for utvikling av grafikk-intensive applikasjoner og WPF har derfor blitt gitt høyest prioritet [..]"
\item Etterfølgeren til Windows Forms . Er ifølge Microsoft erstatteren til teknologien som blir brukt i Sono Flex \cite{User1111:online}. 
\item God støtte for animasjoner. WPF tilbyr blant annet et eget timing system for å enkelt holde tiden når en animerer grafiske elementer \cite{Anima7:online}.
\end{itemize}




 
\subsection{Windows Presentation Foundation} 
 
Ifølge Adam Nathan \cite[p.~9]{WPFbook}, programvare arkitekt hos Microsoft, ble prosessen med å lage WPF igangsatt fordi at mens grafisk maskinvare hele tiden har blitt bedre og billigere - samt at forbrukerens forventninger har fortsatt å stige, så var ingen som hadde adressert vanskeligheten med å lage moderne brukergrensesnitt.  

Han argumenterer med at det fantes utviklere på tiden som for eksempel brukte bitmap bilder til for å få et et annet utseende på for eksempel knappene enn det standardknappene hadde. Problemet er at disse formene for tilpasning ikke bare kan være ressurskrevende å utvikle, men også gi en dårligere brukeropplevelse. 

Han forteller videre at de ønsket å lage et rammeverk som hadde produktiviteten som folk likte med Windows Forms og som var enkel som HTML og Javascript.

WPF gir utvikleren mulighet til å utvikle applikasjoner med å bruke både markup språk og kode. Markup språket brukes vanligvis til å implementere utseende til applikasjonen, mens programmeringspråket brukes vanligvis til å implementere oppførselen. Det vil si at det er mulig å bruke de på tvers, men WPF legger opp til å skille mellom utseende og oppførsel \cite{Intro8:online}. Dette skillet skal ifølge utviklerene gi flere fordeler som blant annet reduksjon av utvikling og vedlikeholdskostnader ettersom utseende-spesifikk markup ikke er tett koblet med oppførsel-spesifikk kode. Det gir også designere mulighet til å jobbe med utseende samtidig som utviklere jobber med oppførsel \cite{Intro8:online}. I WPF er eXtensible Application Markup Language (XAML) brukt som markup språk. Mens man som programmeringspråk kan velge mellom C-Sharp eller Visual Basic.

\subsubsection{C-Sharp} 

Valget av programmeringsspråk sto mellom C-Sharp og Visual Basic (VB), to språk med svært forskjellig syntaks og historie \cite{Compa6:online}. C# har basert syntaksen sin på programmeringsspråket C , mens VB har sine røtter fra programmeringspråket BASIC \cite{Visua3:online} \cite{About8:online}. Begge språkene har de samme mulighetene, forskjellen ligger hovedsaklig i syntaks, så valget av programmeringsspråk ble basert på preferanser \cite{What0:online}. I dette tilfelle ble det valgt å ta i bruk C-Sharp, mye på grunn av likheten med Java som vi hadde erfaring med fra tidligere, men også på grunn dette var språket som ble mest brukt i eksempler og i dokumentasjonen til øyesporingenheten.


\subsubsection{eXtensible Application Markup Language (XAML)} 

XAML, en dialekt av XML, og har vært en viktig del av WPF siden dens introduksjonen i 2006 \cite[p.~17]{WPFbook}. Vanligvis brukes XAML til å spesifisere brukeregrensesnitt objekter og organiseringen av dem, men kan også brukes til andre ting som for eksempel animasjoner \cite{Story5:online}. 

Grunnen til at XAML har blitt brukt er fordi det er enkelt for programmere å samarbeide med eksperter fra andre felt. Nathan forteller at XAML blir felles språket for alle partiene, hovedsaklig gjennom utviklingsverktøy og felt-spesifikke design verktøy. Men også fordi XAML (og XML generelt) er enkelt å lese og forstå \cite[p.~17]{WPFbook}. Figur \ref{listing:knapp} viser hvor enkelt en kan genere en blå knapp som også er lettleselig .


\begin{listing}[ht] 
\inputminted[fontsize=\footnotesize, frame=lines,framesep=2mm,baselinestretch=1.2,bgcolor=lightgray,linenos]{xml}{Code/xamlexample.xml} 
\caption{Hvordan en blå knapp blir definert i XAML} 
\label{listing:knapp} 
\end{listing} 
 
   
\section{IDE og versjonskontroll} 

For å automatisere mye av utviklingen ble det valgt å bruke utviklingsmiljøet Visual Studio 2013 \cite{2013-1:online}. Grunnen til at Visual Studio ble valgt er det for at i tillegg til å ha kode-redigeringsverktøy og debugger, så har den også en WPF designer\cite{What8:online}. Denne designeren tilbyr flere hjelpemidler for å effektivisere prosessen med å lage brukergrensesnitt. Blant annet kan man velge mellom å kode de forskjellige elementene eller man kan dra dem inn fra et sidepanel. Den viktigste er derimot design vinduet, som til enhver tid viser hvordan brukergrensesnitt blir uten å måtte kjøre koden\cite{WPF D1:online}. 
 
For versjonskontroll ble det på grunn av erfaringsmessige årsaker brukt git\cite{AboutGit:online}. Til å ta backup av git filene ble den web-baserte tjenesten Bitbucket brukt. Grunnen til dette var at mye av koden fra Tobii Dynavox var konfidensiell og Bitbucket tilbyr gratis hemmelig oppbevaring. 
 
 
\section{Model View ViewModel} 

En viktig del av oppgaven var å bygge en prototype som Tobii Dynavox kunne bygge og teste videre på. Det var derfor en forutsetning at kildekoden var tilrettelagt for vedlikehold og videreutvikling. For å gjennomføre dette valgte vi å følge arkitekt mønsteret Model view Viewmodel (MVVM). Grunnen til at vi valgte dette fremfor mønster som for eksempel Model View Contoller(MVC)\{MVC a2:online} og Model-View-Presenter(MVP)/cite{The M4:online}. Er fordi MVVM ble utviklet av Microsoft spesifikk for å utnytte funksjonaliteten til WPF \cite{THEM6:online}. Artikkelforfatteren forteller videre at han ser på MVVM som skreddersydd for WPF. 

Det ble tidligere nevnt a WPF legger opp til å skille mellom utseende og oppførsel, MVVM skal hjelpe utviklere å følge dette prinsippet. Ved å skille mellom applikasjons logikk og brukergrensesnitt skal det ifølge Microsoft \cite{Im1online}, gjøre det enklere å teste, vedlikeholde og videreutvikle. MVVM gjør dette ved å dele opp koden i 3 deler - Modeller, ViewModel og View.
Figur \ref{fig:mvvm} viser de tre MVVM klassene view, view-model og model, og interaksjonen mellom disse.
 
\begin{figure}[ht!] 
\centering 
\includegraphics[width=100mm]{Mvvm} 
\caption{Illustrasjon som viser tre MVVM klasser og hvordan interaksjonen mellom dem\cite{Im1online}} 
\label{fig:mvvm} 
\end{figure}


\subsection{Modell}

Modell delen av MVVM står for en av de viktigste delene i enhver applikasjon, nemlig data og informasjon. Modellen sin eneste jobb er å representere og holde dataene som applikasjonen skal bruke. Den skal ikke hente data fra en database eller manipulere data på noen som helst måte \cite{Model7:online}. Dette går under forretningslogikk og skal ifølge mønsteret holdes separat fra modell-klasser. 


\subsection{View}
 
 Som navnet impliserer, skal klasser definert som View stå for visuelle elementer som vinduer, knapper, tekst, farger og organiseringen av dem. Med andre ord, selve brukergrensesnittet \cite{THEM6:online}. Grafiske objektet kan skrives ved hjelp av kode, men i MVVM skal alt som har med View og gjøre skrives XAML.
 
 

 
 
 For at en bruker skal kunne utføre kommandoer og hente data gjennom et view, brukes det i WPF, databindings uttrykk som blir evaluert opp mot viewets gitte datakontekst. I MVVM vil datakonteksten være satt til viewmodellen. Det vil si som figur \ref{fig:mvvm} viser, at alle former for kommandoer, notfikasjoner og databinding skjer gjennom viewmodellen. Et view vil typisk kun forholde seg til en viewmodel, altså et en-til-en relasjon\cite{.   
 
\subsection{ViewModell}
 
 ViewModellen er en nøkkelbrikke i MVVM 
 

The viewmodel is a key piece of the triad because it introduces Presentation Separation, or the concept of keeping the nuances of the view separate from the model. Instead of making the model aware of the user’s view of a date, so that it converts the date to the display format, the model simply holds the data, the view simply holds the formatted date, and the controller acts as the liaison between the two. The controller might take input from the view and place it on the model, or it might interact with a service to retrieve the model, then translate properties and place it on the view.

The viewmodel also exposes methods, commands, and other points that help maintain the state of the view, manipulate the model as the result of actions on the view, and trigger events in the view itself.

MVVM, while it evolved “behind the scenes” for quite some time, was introduced to the public in 2005 via Microsoft’s John Gossman blog post about Avalon (the code name for Windows Presentation Foundation, or WPF). The blog post is entitled, Introduction to Model/View/ViewModel pattern for building WPF Apps and generated quite a stir judging from the comments as people wrapped their brains around it.

I’ve heard MVVM described as an implementation of Presentation Model designed specifically for WPF (and later, Silverlight).

The examples of the pattern often focus on XAML for the view definition and data-binding for commands and properties. These are more implementation details of the pattern rather than intrinsic to the pattern itself, which is why I offset data-binding with a different color:
 
Mens viewet bestemmer hvordan applikasjonen skal se ut, så bestemmer viewmodellen hvordan funksjonaliteten skal være \cite{THEM6:online}. Det er i viewmodellen at egenskaper og kommandoer som viewet kan binde seg opp mot er implementert. Når data  da endres vil view bli varslet og oppdatert deretter(Notified i figur \ref{fig:mvvm}) . På samme måte vil viewet kalle på viewmodellen om at en kommando må kjøres hvis for eksempel en bruker trykker på en knapp. Det er også Viewmodellen sitt ansvar og koordinere interaksjoner med viewet med de modellene som trengs. Det vil si at viewmodellen kan eksponere modellen direkte til viewets slik at de kan bindingen kan skje direkte til dem. Samtidig kan viewmodellen også manipulere data fra modellen , som for eksempel å kombinere to verdier. Eksempel kan være å sette sammen fornavn og etternavn fra en Person modell. Mens forholdet mellom View og Viewmodellen typisk er en-til-en, så vil ViewModellen ha en en-til-mange relasjon.  
 
 
\subsection{MVVM light} 
 
 
//Siden ny utvikling og lage en testplattform, viktig at kode var optimalisert 
//Mvvm passet best skille mellom logikk og blablabla 
//Vikitg å få med design time data, side applikasjonen er svært design tung 
//Finn fordeler 
//mvvm light  - IOC container, design data. 
//Arkitektur 
 
 
 
\section{Utvikling}  
 
 
I seksjon \ref{sec:utgangspunk} ble teknologien som ble brukt til å utvikle prototypen diskutert. I denne seksjonen vil arkitektturen og de viktigste utviklingsdetaljene  bli drøftet.  
 
 
\subsection{Arkitektur} 
 
 
Figur \ref{fig:arkitektur} viser et forenklet bilde over hvordan klassene i kodebasen er koblet sammen. Med forenklet så menes det at flere hjelpeklasser og tredjeparts bilioteker er med i diagrammet. I tilegg så har ikke Views blitt tatt med i diagrammet, men for hver ViewModel så er det View som representerer det.  
 

\begin{figure}[ht] 
\centering 
\includegraphics[width=140mm]{Arkitektur} 
\caption{Figur som viser et overordnet bilde av hvordan de ulike klassene henger sammen} 
\label{fig:arkitektur} 
\end{figure} 
 

 
ApplicationViewModel er vinduet som åpnes når programmet startes. Denne klassen fungerer som en kontainer for applikasjonen og bestemmer kun applikasjonsvinduets størrelse og navigasjon mellom de andre viewmodellene Keyboard, Welcome og UserSettings.  
 
 

\begin{listing}[ht] 
\inputminted[fontsize=\footnotesize, frame=lines,framesep=2mm,baselinestretch=1.2,bgcolor=lightgray,linenos]{xml}{Code/ApplicationContainer.xml} 
\caption{Utdrag fra kode som viser hvordan kontainer er satt opp} 
\label{listing:Kontainer} 
\end{listing} 
 
 
\begin{listing}[ht] 
\inputminted[fontsize=\footnotesize, frame=lines,framesep=2mm,baselinestretch=1.2,bgcolor=lightgray,linenos]{csharp}{Code/CurrentApplicationView.cs} 
\caption{Utdrag fra kode som viser hvordan kontainer er satt opp} 
\label{listing:CurrentAppView} 
\end{listing} 
 
 
 
Figur \ref{listing:Kontainer} viser hvordan ContentControl elementet i ApplicationView fungerer som en kontainer ved at innholdet er bundet til egenskapen CurrentViewModelBase. Som vil si at innholdet i ContentControl vil basere seg på hva som er satt som CurrentViewModelBase i ApplicationViewModel. Figur\ref{listing:CurrentAppView} viser hvordan denne egenskapen er implementert i Viewmodellen. Utfra koden så kan man se at til forskjell fra viewet, som hadde en referanse til egenskapen, så er det ingen direkte referanse fra viewmodellen til viewet. Dette er for å følge prinsippet til MVVM om at viewmodellen ikke skal ha noe kjennskap til Viewet, som gjør koden løs koblet(les: loose coupled). Men siden dette er en verdi som antakeligvis vil endre seg i løpet av kjøretiden er det nødvendig at Viewet blir gjort oppmerksom på forandring. Dette skjer ved å avfyre RaisePropertyChanged, dette kallet vil varsle rammeverket om at en endring har skjedd. Når da Viewet har en binding til denne egenskapen vil han bli oppmerksom på endringen og oppdatere deretter \cite{MVVM4:online}. 
 
CurrentViewModelBase er av typen ApplicationViewModelBase som vil si at det som er i kontaineren må være av nettopp denne typen. ApplicationViewModelBase er som man ser utifra figur  en abstrakt og har kun en abstrakt metode for å hente navn.  Det vil si at de klassene som arver fra ApplicationViewModel må implementere metoden. Fra figur kan man se at de klassene som arver fra ApplicationViewModel og med det har mulighet til å være i kontaineren er, KeyboardViewModel, UserSettingsViewModel og WelcomeViewModel. Sammen representerer disse hoved funksjonaliteten til programvaren. KeyboardViewModel er hoveddelen av applikasjonen, det her en bruker har mulighet til å skrive setninger med å trykke på de ulike symbolene for så å gjøre dem om til tale. I UserSettings kan en bruker se og endre på de ulike innstillingene. Mens \texttt{WelcomeViewModel} er kun laget for testingen og funksjonaliteten er begrenset til at en bruker kan velge alder og kalibrere øyesporingsenheten. 
 
 
 \subsection{Skrivebordet}
 
KeyboardViewModel og KeyboardView utgjør sammen det som blir kalt skrivebordet. Skrivebordet er den delen av programvaren som tilbyr hovedfunksjonaliteten, å la en bruker kunne skrive setninger. For å gjøre dette så må er det en del viktig elementer som må være tilstede. Man kan si at skrivebordet skal være en digital representasjon av det klassiske tastaturet. Der det er ord med symboler istedenfor bokstaver, og trykk skjer ikke ved fysisk trykk, men ved å se på knappen over en periode. Utenom dette skal funksjonaliteten være mye den samme. De mest nødvendige funksjonene som å kunne viske og bygge setninger må ihvertfall være tilstede. I tillegg må prototypen ha mulighet for å kunne presentere setningene som lyd. 

\begin{figure}[ht!] 
\centering 
\includegraphics[width=100mm]{skrivebord} 
\caption{Skjermdump av skrivebordet i prototypen} 
\label{fig:skrivebord} 
\end{figure} 
 

\subsubsection{Innlesning}


For å kunne skrive med skrivebordet er det nødvendig at det er fullt opp med ord og symboler. For å gjøre dette blir metoden texttt{GetInitialPages} vist i figur .. kalt fra konstruktøren. Denne metoden kaller så igjen på metoden GetCategory("Frontpage") i klassen \texttt{Dataservice} vist i figur ... Grunnen til at ViewModellen ikke kun direkte henter symbolene er fordi de er lagret på et tekstdokument. Så for å separere data fra presentasjon- og forretningslogikk så brukes det et data tilgang lag i mellom. For å gjøre om tekstfilen til strukturerte data bruke datatjenesten klassen TextParser til å tolke den. 


\subsubsection{Data og innlesning av dem}

Programvaren skal strebe etter å kunne tilby brukeren ordene han har lyst å bruke. Noe som kan være utfordrende med tanke på at et barns vokabular allerede i alder av 5 år kan være opp i mot XXXX ord. Med tanke på hvilke ord som er i vokabularet vil variere  fra person til person så må programvaren ha en god del mer enn dette. I tillegg må også programvaren tilby et bilde for vært av ordene. 

En naiv fremgangsmåte for å lage brikkene er å statisk initialisere hver brikke med et ord og bilde. Problemet med dette er først og fremst at XAML koden blir unektelig lang som følge av at en må skrive en knapp for vært tilfelle. Den at en også må inn i koden for å gjøre endringer eller legge til nye ord, er ugunstig ettersom det er en ressurskrevende prosess. Løsningen på dette var å oppbevare alle ordene i en separat fil for å skille mellom brukergrensesnitt og data. For at maskinen skulle ha mulighet til å lese dataene ble de lagret i JavaScript Object Notation(JSON) format. JSON er ifølge sine egne nettsider \cite{JSON7:online} et lettvekt data-utvekslings format. Som er lett for mennesker å lese og skrive og det er lett form maskiner å tolke og generere. Kode \ref{listing:jsonfile} viser et utdrag fra json filen hvor ordene og stien til bildet som representerer ordet, er lagret. Filen består i en liste med JSON objekter som har attributtene Name og Image. Der Name er ordet og image er stien til symbolet. Dette er bare et lite utdrag fra filen og viser kun det som ville tilsvart fire brikker i programvaren. Det første objektet "I" er et ord, mens de tre neste er kategorier. Hver kategori består av flere ord. Ordene som tilhører en kategori er lagret i en egen fil i en mappe med samme navn. Figur \ref{jsonstructure} viser strukturen på filene, der hver kategori har sin egen mappe. Dette gjør at data hentes "just-in-time". Det vil si at istedenfor at ordene ligger i minnet til enhver tid, så hentes de kun når ved behov. Eksempelvis hvis en bruker trykker på kategorien "Food and Drink" så vil ordene hentes fra "FoodAndDrink.json" i mappen "FoodAndDrink". Noe som sparer på minnet, men som kan forringe kjøretiden. For når det er snakk om så store mengder data som alle ordene ville gitt, er det ikke gunstig å bevare dem i minnet. 


\begin{listing}[ht] 
\inputminted[fontsize=\footnotesize, frame=lines,framesep=2mm,baselinestretch=1.2,bgcolor=lightgray,linenos]{json}{Code/JSONfile.json} 
\caption{Utdrag fra filen som inneholder ord og sti til bilde som representerer det i JSON format} 
\label{listing:jsonfile} 
\end{listing} 
 
 
 \begin{figure}[ht!] 
\centering 
\includegraphics[width=100mm]{JsonStructure} 
\caption{Bilde av dokumentet hvor de ulike symbolene er registrert} 
\label{fig:jsonstructure} 
\end{figure} 


For å kunne ta i bruk objektene definert i tekstfilen må de tolkes fra JSON format til .NET objekter. Denne prosessen, å trekke ut datastrukturer fra bytes, kalles deserialisering. For å gjøre dette har vi brukt et tredjeparts rammevekt kalt JSON.net \cite{Json.0:online}. Det finnes en innebygd funksjon i .NET kalt DataContractJsonSerializer \cite{DataC3:online} som også greier å deserialisere et JSON dokument. Grunnen til at JSON.NET ble brukt er fordi i dette biblioteket er det mulighet for å automatisk deserialisere en helt liste av JSON objekter til .NET liste uten å manuelt måtte legge inn objektene. Ifølge utviklerens egne nettsider skal rammeverket være 50 prosent kjappere enn DataContractJsonSerializer, men ytelsen er avhengig av hvilke datasett som brukes, og har ikke vært en faktor i avgjørelsen. 


\begin{listing}[ht] 
\inputminted[fontsize=\footnotesize, frame=lines,framesep=2mm,baselinestretch=1.2,bgcolor=lightgray,linenos]{csharp}{Code/JSONparser.cs} 
\caption{Koden som konverterer JSON filen til en IList} 
\label{listing:JsonParser} 
\end{listing} 


Koden i figur \ref{listing:JsonParser} deserialiserer JSON dokumentet om til en dataliste. Fra linje 5 til 12 skrives teksten fra filen over til en streng som kan manipuleres i koden. På linje nummer 14 blir strengen konvertert til et dataobjekt. Det at dokumentet blir gjort om fra streng til et komplekst objekt på kun en linje viser noe av styrken til json.NET. 

\begin{listing}[ht] 
\inputminted[fontsize=\footnotesize, frame=lines,framesep=2mm,baselinestretch=1.2,bgcolor=lightgray,linenos]{csharp}{Code/CategoryModel.cs} 
\caption{Category modellen har egenskapen navn og en liste over alle sidene som utgjør alle symbolene som hører til i kategorien} 
\label{listing:CategoryModel} 
\end{listing} 


\subsection{Teksttolker}

Under utviklingen ble det bare lagt inn tilfeldige ord i prototypen, men ettersom prototypen begynte å bli ferdigstilt var det behov for et større utvalg ord og symboler Det ble derfor gitt tillatelse av Tobii Dynavox til å ta i bruk det samme bildene som de bruker i deres programvare. Denne bildepakken heter SymbolStix og har gir tilgang på cirka 16 000 symboler og er levert av et eksternt selskap som heter n2y \cite{n2y}. 

\begin{itemize}
\label{itm:egenskaper}
\item Symbol ID
\item Dato oppdatert
\item Dato laget
\item Kategori Navn 
\item Filnavn
\item Filsti
\item Ord
\item Synonymer
\item Tysk, nederlandsk, norsk, svensk, dansk, engelsk, spansk, italiensk, fransk, portugisisk.
\end{itemize}


Sammen med bildepakken fulgte det med et dokument som hadde en beskrivelse av vært symbol. For vært symbol var egenskapene beskrevet i liste \ref{itm:egenskaper} tilgjengelig. Disse var strukturert med at de kom i rekkefølge og hadde tilde notasjon(tilde) for å skille mellom dem. Dette gjorde at det var mulig å implementere en løsning for å gjøre dem om til datastrukturer. I tillegg til å tilby det samme som JSON dokumentet, ord og bildesti, så har den også ordets kategori og synonymer for ordet. Det er også mulighet for å få ordet på flere språk og synonymer til de ulike språkene. Noe som åpnet for flere muligheter, men for å få til dette, måtte det legges til en annen form for innlesing av data ettersom dokumentet ikke var lagret i JSON format og I motsetning til å dele opp symbolene opp i separate dokumenter basert på hvilken kategori de tilhørte, så var alle deklarert på et dokument. Så for å kunne løse dette sto vi mellom å lage et skript som oversatte dokumentet til JSON format eller lage en alternativ innlesning. Å omgjøre det fra tekst til JSON hadde vært mulig, problemet med dette er at dokumentet utsatt for oppdatering. Så hver gang dokumentet blir forandret må oversetting gjøres på ny. Så det ble da bestemt for å implementere en teksttolker. 

Tekstolkeren er er mer kompleks enn JSON tolkeren. Hovedsaklig fordi det ikke finnes noe biliotek som automatisk tilegner egenskaper til symbolene. Egenskapene blir derfor gitt til vært symbol etterhvert som de blir lest inn. En annen utfordring var som tidligere nevnt at alt befant seg på et dokument som da totalt utgjorde 16 000 linjer med beskrivelse av symbolene. Slik at hvis en skulle ha brukt samme fremgangsmåte som med JSON parseren å kun lest inn for den gitte kategorien. Så hadde en i verste fall måtte ha lest 16 000 linjer med tekst. Dette er ikke gunstig. Dokumentet har flere kategorier og symboler som ikke er nødvendig for barn, blant annet egne kategorier som eksempel Brasil, hebraisk og sveitsisk. Det er også flere kategorier som kunne vært greit å ha, men som ikke prioriteres som blant annet kategorier om kjendiser og en egen om amerikanske byer. Det ble derfor bestemt å kun velge kategorier som var nødvendige.

\begin{figure}[ht!] 
\centering 
\includegraphics[width=100mm]{datafil} 
\caption{Bilde av dokumentet hvor de ulike symbolene er registrert} 
\label{fig:dok} 
\end{figure} 


\subsection{Modellene}

Modellene i MVVM mønsteret enkapsulere forretningslogikk og data. Der forretninslogikk er definert som all applikasjonslogikk med tanke på innhenting og håndtering av applikasjonsdata for forsikre seg om at data er konsistent og validering er pålagt. Bruk av modeller skal hjelpe til med å maksimere gjenbruk.

I applikasjonen er det flere modeller, men det som blir brukt mest er Symbol modellen. Denne representerer alle data i symboltabellen. Der det blant annet er et bilde og et ord som representerer dette bildet. Den har også egenskaper som forteller om hvilken type knapp det er, altså navigering,kategori eller ord. Hvilken knapp det er vil påvirke hva som skjer når en bruker interagere med denne.



\section{SymbolStix} 
 
En viktig del av programvaren er å hjelpe barn å bygge et vokabular ved å visualisere ord med bilder. Det vil si at for hvert ord må det finnes et bilde, noe som kan bli svært mange. Tobii Sono Flex har løst denne oppgaven med å bruke et kommersielt tredjeparts bildebibliotek kalt SymbolStix \cite{n2y}, utviklet av n2y. Bildene i biblioteket består av svært enkle tegninger som kun har med det mest nødvendige får å få frem konseptet eller ordet symbolet prøver å representere. Det at Sono Flex og at dette bildebiblioteket var tilgjengelig gjennom Tobii gjorde at den samme bildepakken skulle bli brukt i prototypen. Dette gjorde også medføre at ikke unødvendig med ressurser ble brukt på anskaffe bilder for hvert ord. 
 
 
\section{Utfordringer}

\subssection{Problem} 
 
Symbolstix bildene er lagret i Enhanced Metafile Fomat (EMF), som er et 32-bit format som kan inneholde både vektor og bitmap informasjon\cite{AboutEMF}. Problemet med EMF formatet er at det ikke finnes støtte for dette i WPF, med den begrunnelse at det har blitt funnet en kritisk sårbarhet\cite{EMFVulnerability} og at formatet er mer mottakelig for sårbarheter\cite{EMFForum} enn andre bildeformater. I Sono Flex fungerer dette fint, fordi det bygget på Windows Forms(WF),  et rammeverk som har innlagt støtte for EMF. WPF har støtte for å bruke WF elementer,  en funksjon som ble lagt inn for å lette overgangen fra WF til WPF. Som igjen gjorde det mulig å hente bilder med EMF format. Ulempen er at man må trekke inn store deler av WF biblioteket noe som fører til en betraktelig økning i størrelse. For å aktivere støtten, deklarerer en WindowsFormHost i xaml filen og denne vil fungere som en kontainer for alle WF elementer. Innenfor denne kontaineren vil alt fungere som i et WF miljø. Ved å gjøre dette ble alle bildene hentes og vist i prototypen.  
 
 
Testing av applikasjonen viste derimot at det ikke lenger var mulig å trykke på knappene, eller mer korrekt, det var ikke mulig å trykke på  bildet der knappen var. Dette viste seg å være kjent problem, og er kjent som Airspace problemet (kilde) og kommer av at alle WF elementer uansett vil legge seg over alle WPF elementer. Et problem som ikke kunne være med i prototypen. En fix til dette ble laget og planlagt lansert i .NET versjon 4.5, men da versjonen ble lansert, var ikke denne fiksen med. Det finnes ulike omveier rundt på problemet, men få tilfredsstillende.  
 
 
Det ble derfor prøvd å konvertere EMF filene til bitmap for så å vise de i applikasjonen. Fordelen med denne fremgangsmåten var at vi slapp å trekke inn windows forms biblioteket. Ulempen var derimot at hver gang et symbol skulle hentes så måtte dette først konverteres til Bitmap for deretter og rendres. En tidkrevende prosess som gjorde at det tok flere sekunder å laste brukergrensesnittet. Det ble derfor prøvd å konvertere alle filene og deretter lagre dem som jpg. Konverteringen ville da kun gjøres en gang. Deretter ville applikasjonen hente jpg filene direkte, uten noe omvei. Siden wpf har innebygd støtte for formatet. Igjen viste det seg at dette ikke var godt nok ettersom jpg filene hadde blitt for store i overgangen fra vektor til raster grafikk og de manglet gjennomsiktighet. For mens EMF filene lå på under 10kb endte alle de konverterte filene opp på over 100kb. Noe som i selv ikke er så stort, men som gir utslag på lastingen av applikasjonen når det kan være opp mot 200 bilder som skal lastes. 
 
 
Den endelig løsningen ble å laste ned et profesjonelt konverteringsverktøy kalt AVS Image converter. En svært ømfintlig prosess som tok svært lang tid, men som gjorde det mulig å konvertere de orginale EMF filene til png og samtidig holde størrelsen på rundt 10 kb. 


\subsection{Sette seg inn i teknologiene}

\subsection{Følge MVVM mønsteret}
 
 
\section{Applikasjonen} 

Prototypen har den samme layouten som Tobii Sono Flex, med en menylinje etterfulgt av en symboltabell under. Det er derimot en del variasjoner inne i hver av disse komponentene. 
 
 
\subsection{Menylinjen} 
 
Menylinjen dekker 1/5 del av applikasjonens vindu og består av 4 knapper og listen over ord som brukeren har trykket på. Denne listen vil heretter bli referert til som ordlisten. 
 
 
\subsubsection{Tilbaketasten} 
 
Tasten som befinner seg til høyre for ordlisten er tilbake tasten. Som på et vanlig tastatur vil et trykk på denne knappen medføre at det siste ordet i ordlisten fjernes. 
 
 
\subsubsection{Ordlisten} 
 
Når en bruker trykker på et ordsymbol så vil dette legge seg i ordlisten og når han trykker på "prate" knappen så vil ordene i listen bli gitt gjennom høyttalerne som naturlig tale og deretter fjernes fra listen. Selv om det er plass til uendelig med ord i listen så vil det kun være mulig for brukeren å se maks 4 om gangen. Hvis det allerede er fire symboler i listen når brukeren trykker på et nytt symbol, så vil de tre første bli "fjernet" mens det fjerde og det nye symbolet vil være igjen. Hvis brukeren igjen fjerner de to siste ordene ved å trykke på tilbaketasten,  vil ordene som kommer før igjen bli presentert for brukeren i setningsliseten.  
 
 
\subsubsection{Hjem} 
Ved å trykke på "hjem" knappen vil brukeren alltid bli ført til førstesiden av applikasjonen uavhengig av hvor han befinner seg.  
 
 
\subsubsection{Innstillinger} 
Hvis brukeren første er på "hjem" siden av applikasjonen så vil knappen byttes ut med en "innstillinger" knapp. Ved å trykke på denne vil det åpnes et nytt vindu hvor brukeren vil ha mulighet til å sette og endre på diverse innstillinger. Detaljene rundt denne siden er beskrevet i seksjon. 
 
 
\begin{figure}[ht!] 
\centering 
\includegraphics[width=100mm]{MenylinjeP} 
\caption{Skjermdump av menylinjen i prototypen} 
\label{fig:menylinjen} 
\end{figure} 
 
 
\section{Symboltabellen} 
 
 
Symboltabellen består av like mange kolonner og rader som den gjør i Sono Flex,  men det er måten den oppfører seg på som er annerledes.   
 
 
\section{Sammendrag} 

Kapittelet forklarer hvordan prototypen fungerer og prosessen med å utvikle den. Prototypen har blitt implementert med en moderne teknologi som fortsatt er støttet. Det har blitt lagt vekt på kodekvalitet under utviklingen. Blant annet så har vi følgt et design mønster som passer bra både til teknologien og plattformen som den skal kjøre på. Alle kravene satt i kravspesifikasjonen har blitt implementert, men noen ikke i like stor grad som ønsket. 


\chapter{Animasjoner, lyd og brukertilpasning}
 
 
 
\section{Animasjon} 
 
 
I denne prototypen skjer animasjonene som en respons på at brukeren trykker på et symbol. Hva som skjer avhenger av hvilken type symbol det er. De ulike typene som finnes i symboltabellen ble forklart i kapittelet om Sono flex, og er ordsymbol, kategorisymbol og navigasjonssymbol. Med Animasjonene vil en først og fremst gi brukeren tilbakemelding om at et valg har blitt gjort, men de skal også gjøre programvaren mer spennende.  I Sono Flex vil ingen av knappene ha noe visuelt som gjør at brukeren kan skille hva som skjer, utenom at symboltabellen bytter ut de eksisterende symbolene. I dette tilfelle vil en se at symbolet går fra å være et statisk symbol til å bli flyttet til setningslisten og bli en del av ønsket setning.  
  
 
 
\subsection{Animasjoner: ordsymbol} 
 

Når en bruker trykker på et ordsymbol, vil symbolet legge seg i setningslisten og er da klar for gjøres om til tydelig tale. Når brukeren har fullført setningen i Sono Flex blir brukeren presentert for endringen ved at symbolet vises i setningsfeltet. Med prototypen ønsker vi  å gi en mer visuell representasjon av denne endringen til brukeren. Ved å la brukeren kunne velge mellom to animasjoner som oppstår når han trykker på ønsket ordsymbol. 

\subsubsection{Glide animasjon}
Den ene animasjonen er vist i figur \ref{fig:OrdAnimasjon} og viser \textit{Glide Animasjonen}. I tilfelle med figuren under har brukeren akkurat trykket på brikken hvor det står "oksefrosk". Det dukker så opp en brikke som er identisk med den han trykket på. Denne brikken begynner så å gli i en rett linje mot målet, som er første posisjon i setningslisten. Hadde det allerede vært en brikke i listen ville den ha tatt høyde for dette og animasjonen tilpasset seg. 
 
 
 \begin{figure}
 \centering
 \includegraphics[height=18cm,keepaspectratio]{Frogs}
 \caption{Langt Bilde}
 \label{fig:OrdAnimasjon}
 \end{figure}


\subsubsection{Minimer - Maksimer}

Den andre animasjonen 

   Ordsymbolet krympes til 5 prosent av størrelsen. 
   I setningslisten vil en kopi av det krympede ordsymbolet dukke opp. 
   Kopien forstørres så til den opprinnelige størrelsen. 
   Ordsymbolet som opprinnelige ble trykket blir forstørret til opprinnelig størrelse. 
 

\subsection{Animasjon: Kategorisymbol} 
 
 
Når en bruker trykker på et kategorisymbol vil symbolene i symboltabellen byttes ut med symbolene som hører til i den gjeldene kategorien. Hvis en trykker på kategorien "dyr" så vil tabellen fylles med ordsymbol som "Hund", "katt", "tiger" o.s.v. Her ønsker animasjonen å hjelpe til med å fortelle at man ikke trykker på ordet dyr, men kategorien dyr og at brukeren får en forståelse for forskjellen mellom dem. Animasjonen startet ved at symbolet som brukeren trykket på forstørres helt til den dekker hele symboltabellen. Det vil si at ingen av de andre symbolene vises, kun kategorisymbolet. Symbolet vil så igjen forminskes, men symbolene i tabellen vil være erstattet av symbolene som tilhører kategorien.  
 
 
 
 
\subsection{Animasjon: Navigasjonssymbol} 
 
 
Den siste typen symbol er navigasjon og ved interaksjon vil den navigere mellom sider når det ikke er plass til alle symbolene på en side. Det vil si at symbolet skal kunne navigere fremover, fra side 1 til 2 og 2 til 3 og bakover igjen. 

Figur \ref{fig:navAni} viser hva som skjer fra brukeren trykker på neste side nede i høyre hjørne. Når brukeren trykker fremover starter animasjonen ved at alle symbolene som er på gjeldene side og neste side begynner å bevege seg mot venstre. Slik at symbolene på gjeldene side kontinuerlig blir byttet ut med de på neste. Symbolene på siden sklir ut, mens symbolene på neste sklir inn og overtar plassene. Når brukeren nå trykker på bakover-symbolet vil det samme skje bare at symbolene sklir mot høyre. 
 

\begin{figure}
\centering
\includegraphics[height=18cm,keepaspectratio]{Slide}
\caption{Smidig glide animasjon}
\label{fig:navAni}
\end{figure}



\section{Lydeffekter} 
 
Som en del av rapporten ønsket vi å undersøke hvilken påvirkning lydeffekter hadde på brukeren. Hovedsakelig var målet å finne ut to ting. I hvilken grad hjelper effektene brukeren i å navigere rundt i applikasjonen og om det vil påvirke brukerens helhets uttrykk av applikasjonen.
 
 
\subsection{Lydeffekter i prototypen} 
 

I prototypen er det lagt til flere lydeffekter, utelukkende i form av små lydklipp på maks 1 sekund. Effektene vil kun bli avspilt som en respons til noe brukeren foretar seg eller for å informere om en hendelse. Denne begrensning finnes for ikke å skape forvirring hos brukeren. Hvis lyd også hadde blitt avspilt tilfeldig så ville meningen med effektene forsvinne, fordi brukeren blir lurt til å tro at programvaren har registrert en interaksjon, selv om han ikke har foretatt seg noe. Det er derfor viktig at det er lett for brukeren å skille mellom effektene som representerer en interaksjon og informasjon. 

De ulike interaksjonslydeffektene er også forskjellige, og hvilken som blir spilt er avhengig av,  som med animasjon,  hvilken type komponent brukeren samhandler med. Eksempelvis vil lyden som blir avspilt når brukeren trykker å tilbaketasten representere noe som forsvinner. De ulike lydene vil ikke bli beskrevet, fordi det fort kunne blitt absurd. 
 
 
 
\section{Brukertilpasning / Innstillinger}

For at en bruker skal kunne tilpasse prototypen etter egne preferanser, må han først trykke på knappen "innstillinger" i menylinjen. Han vil da bli navigert  til siden vist i figur \ref{fig:fart}, hvor han kan tilpasse farten til animasjonene som er brukt i prototypen. På siden vil de blå sirklene bevege seg med den hastighet de representerer, slik at brukeren får et inntrykk av farten. Fra figuren kan man se at det foreløpig kun er seks elementer som kan tilpasses.


\begin{figure}
\centering
\includegraphics[width=16cm,keepaspectratio]{Fart}
\caption{Animasjonsfart innstillinger}
\label{fig:fart}
\end{figure}

En annen innstilling som brukeren kan endre på, er fargepaletten, kalt tema. I figur \ref{fig:my_label1} og \ref{fig:my_label2} vises de to temaene som er implementert i prototypen. Et hvor fargene er hentet fra tv-serien "powerpuff" og et fra serien "Turtles". Disse kan ikke være med i et sluttprodukt på grunn av opphavsrettigheter, men viser hvordan det kan løses. Ved å velge et av temaene vil fargene i prototypen endres deretter.

\begin{figure}
\centering
\includegraphics[width=16cm,keepaspectratio]{turtles1}
\caption{Ninja Turtles fargepalett }
\label{fig:my_label1}
\end{figure}

\begin{figure}
\centering
\includegraphics[width=16cm,keepaspectratio]{powerpuff1}
\caption{Powerpuff fargepalett }
\label{fig:my_label2}
\end{figure}


Når en bruker har endret på innstillinger er det greit at disse holder seg slik neste gang applikasjonen kjøres. For å holde innstillingene mellom kjøringer brukes en .NET tjeneste kalt Brukerinnstillinger \cite{Using812:online}. Dette gjør at ulike brukere på samme maskin kan ha forskjellige innstillinger, fordi den baserer seg på bruker og ikke på applikasjon.


\input{Chapters/7Testing}
\chapter{Konklusjon}

\section{Konklusjon}

Oppgaven har bestått av to delmål. Det første var å utvikle en high-fidelity prototype. Det andre var å implementere ulike animasjoner og legge til lyder for å se hvilken innvirkning dette hadde på den tiltenkte målgruppen. Vi skulle også gjøre det mulig for en bruker å  tilpasse programmet ved øyesporing. 

1) Vi har utviklet en high-fidelity prototype som lar en bruker sette sammen enkle setninger. Den er kompatibel med en øyesporingsenhet som gjør at en bruker kan navigere ved å kun bruke øyene. En som tar i bruke denne interaksjonsformen har tilgang på alle de samme funksjonene som en datamus bruker. Når en bruker har skrevet en setning har mulighet for å presentere dette som lyd. Ettersom programmet er tiltenkt barn, er vært ord representert med et symbol. En utvikler kan enkelt bytte ut eller legge til nye ord og symboler ved å legge de inn i den medfølgende tekstfilen. Programmet er bygget på en moderne teknologi og har en kodekvalitet som gjør at den kan brukes i av andre i fremtidige prosjektet. Det er også mulighet for å videreutvikle prototypen til en ferdig programvare. 

2) I prototypen har vi lagt til flere animasjoner og lyder for å forbedre brukervennligheten. Vi har implementert ulike animasjoner og lagt til forskjellige lyder for differensiere mellom de ulike brikkene. Vi har også gjort det mulig for en bruker å tilpasse ulike aspekter i programmet etter egne ønsker. Dette er også mulig ved bruk av øyesporing. 

For å finne ut hvilken påvirkning dette hadde å brukerene ble det foretatt testing. Vi fikk ikke til å rekruttere folk som bruker Sono Flex eller lignende systemer til dagligdags. Dette gjorde at testen ble foretatt på barn i samme aldersgruppe på målgruppen. Testingen viste indikasjoner på at animasjoner og lyd gir brukeren en bedre forståelse av programvaren. Men får å kunne konkludere må det mer omfattende testing til.


\section{Videre arbeid}

\textbf{Omfattende testing}
I oppgaven har vi hatt vanskeligheter med å rekruttere testere. Først og fremst ble det ikke mulighet for å teste på målgruppen, altså barn i alderen 6 år som dagligdags bruker programvare til kommunikasjon. Og de deltakerne vi fikk, var ikke nok til å konkludere. Det vi derimot fikk, var vage indikasjoner. Så for å få bekrefte eller avkrefte disse, må det mer omfattende testing til. 


\textbf{Internasjonalisering}

\textbf{Legg til talespråk}
Foreløpig er naturlig tale kun tilgjengelig på engelsk. Det er enkelt å legge til flere språk i koden så lenge de er installert på maskinen. 

\textbf{Legg til flere innstillinger som  bruker kan endre}
Det er mulig for en bruker å tilpasse animasjonsfart, symbolstørrelse og temaer. Det burde derimot være mulighet for en bruker å endre mye mer ved kun øyestyring. Som for eksempel mulighet til å designe egne kategorier og flytte symboler. 

\textbf{Hva som skal til for videre bruk} 
Den viktigste funksjonaliteten er tilgjengelig i prototypen og den har blitt testet på barn og studenter. Men det er fortsatt mye som må gjøres for at den kan sies å være ferdig. Blant annet så har Sono Flex mye funksjonalitet som fortsatt ikke er implementert i prototypen. Den har blitt testet, men ikke på brukere som faktisk har behov for den. Så videre arbeid for å ferdigstille prototypen bør fokusere på implementasjon av manglende funksjonalitet og testing på målgruppen.

\textbf{Kategorisering og organisering}
En del av oppgaven var å se på optimal kategorisering og organisering. Dette har det desverre ikke blitt tid til, ettersom dette viste seg å være mer omfattende en først tiltenkt. Men for at et barn effektivt skal kunne bruke Sono Flex eller lignende programvare mer effektivt, bør dette feltet utforskes mer.


Mye av denne oppgaven er basert på videre arbeid, eller legge til rette for videre arbeid. Det kan sies at vi har laget en plattform for videre arbeid som både Tobii og fremtidige studenter kan forske på.


Prototypen inneholder funksjonalitet som gjør at den fungerer, den har blitt testet og både på barn og studentet, med og uten øyesporing. Allikevel vil det alltid dukke opp bugs og fail og v

Brukertilpasning er det mye som mangler

Lag tester til koden. 

Videre testing av de allerede implementere animasjonene

Hva skal til for at det er en ferdig programvare.
    - Legge til flere brikker, dette er enkelt.
    - Sono flex har mange flere funksjoner som ikke er tilgjengelig. Disse må implementeres. 
    - Mye mer testing.
    - 


Som en oppsummering så kan en si at med så lite testpersoner så kan man ikke si noe sikkert, men vi kan konkludere med at prototypen fungerer og at et barn i seks år alderen enkelt greier å uttrykke seg via denne. I tillegg er det indikasjoner på at animasjoner gjør at en ny bruker av programvaren bruker lengre tid enn uten animasjoner. Det er også indikasjoner på animasjoner og lyd gir brukeren en bedre forståelse for hva de ulike typene symbol representerer.  

For å utforske mulige løsninger som kan øke brukervennligheten til Sono Flex  vil en rekke funksjoner og design implementeres og testes. 

Hovedsakelig skal vi se på hvordan ulike elementer kan hjelpe en bruker i å finne ønsket symbol raskere. For eksempel hva vil gjøre at et barn enklere husker hvilken kategori han må trykke på for finne ønsket symbol. For å gjøre dette vil fokuset ligge på audiovisuelle hjelpemidler, men det vil også ses på hvordan en bør organisere de forskjellige ordene nedenfor. Listen nedenfor gir en  beskrivelse av det vi ønsker å utforske.



fungerer sammen med en øyesporingsenhet og mus, slik at en bruker kan velge alle funksjoner som er tilgjengelig for mus er også tilgjengelig ved åp kun bruke øyet.

Programmet er bygget på en moderne teknologi som gjør at den enkelt kan brukes i fremitdige prosjekter blant annet for masterstudenter eller for selskapet. Det er også mulighet for å ferdigstille den til et ferdig produkt.-



Vi har utviklet en prototype som lar barn sette sammen enkle setninger kun ved hjelp av øyene. Prototypen har kommet som et resultat av at den eksisterende løsningen, Sono Flex, er implementert i en teknologi ikke var kvalifisert for det vi ønsket å utforske i andre del. En stor del av oppgaven ble derfor å utvikle denne fra bunn av. Tobii så også muligheten til at dette kunne bli en plattform for fremtidige oppgaver og muligens starten på en ferdig programvare. 
























Den endelige koden er derfor av god kvalitet og det skal være mulig for andre utviklere å forstå og dermed videreutvikle. 


2) I prototypen har vi implementert de fleste komponentene som vi på forhånd ønsket å utforske. 


//Utviklet animasjoner og lyder


//Lar barn bruke øyesporing til å kommunisere, snakker med øyesporingsenheten.
//Koden har ved bruk av mønster, kodekonvesjoner og generelt betenksom rundt koding holdt en god kode.
//Tar i bruk de eksisterende symbolene.
//enkelt å legge til nye symboler.




Tobii Dynavox ønsket teste programvaren på en ny teknologi 

mulige forbedringer på en av deres programvarer. I dette tilfelle Sono Flex. Problemet var at den teknologien som denne er implementer i, holder på å bli utfaset. Derfor ble oppgaven todelt. Eller ret


Oppgaven gikk opprinnelig kun ut på å utforske mulige forbedringer vi kunne gjøre på den programvaren Sono Flex. 


Målet med denne oppgaven var todelt, først måtte vil utvikle en high-fidelity prototype med å


Målet med oppgaven har derfor blitt todelt. Det vil si at i tillegg til å bygge ut funksjoner og teste disse på den eksisterende programvaren, vil også programvaren som disse skal fungere på utvikles. Første delmål av oppgaven vil derfor være å utvikle programvaren. Andre delmål vil være å teste ulike design og funksjoner.


I oppgaven har vi gjort to ting, vi har laget en high-fidelity prototype 































I dette kapittelet vil vi først evaluere de to delmålene som er presentert i det første kapittelet. Her vil vi også komme med forslag til videre arbeid. Til slutt vil det være en konklusjon.

\section{Prototype}

Første del av oppgaven var å utvikle en high-fidelity prototype med fokus på hovedfunksjonaliteten til Sono Flex. Som vil si at den ligner svært mye et ferdig produkt med mye detaljer og funksjonalitet. Dette gjør at en kan foreta konklusjoner om oppførselen til det ferdige produktet. Grunnen til at det måtte utvikles en prototype istedenfor å videreutvikle på den eksisterende var:

- Teknologien som den eksisterende programvare Sono Flex er bygget på, er ikke tilrettelagt for animasjoner. Noe som var en viktig del av oppgaven.

- Den er også blitt satt i vedlikeholds modus av utvikleren Microsoft. Som vil si at feil og bugs vil bli rettet opp, men at det vil ikke komme nye funksjonalitet. Et tegn som kan tyde på at den fases ut.

Prototypen ble utviklet på WPF en teknologi som har eksistert lenge nok til at den er moden samtidig som den fortsatt blir utviklet av Microsoft. Teknologien ble valgt 







//Valget falt på WPF ligger tett opp til den eksisterende teknologien, men er bedre Støtte for hardware

//Kodekvalitet, fulgt kodestandarder, mønster som passer til teknologi. 


a prototype that is quite close to the final product, with lots of detail and functionality. From a user testing point of view, a high-fidelity prototype is close enough to a final product to be able to examine usability questions in detail and make strong conclusions about how behavior will relate to use of the final product.
\includepdf[pages={1,2}]{Appendix/Foresporsel.pdf}



\medskip

\bibliographystyle{chicago}%Used BibTeX style is unsrt
\bibliography{sample}

\end{document}
