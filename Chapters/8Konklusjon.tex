\chapter{Konklusjon}

\section{Konklusjon}

Oppgaven har bestått av to delmål. Det første var å utvikle en high-fidelity prototype. Det andre var å implementere ulike animasjoner og legge til lyder for å se hvilken innvirkning dette hadde på den tiltenkte målgruppen. Vi skulle også gjøre det mulig for en bruker å  tilpasse programmet ved øyesporing. 

1) Vi har utviklet en high-fidelity prototype som lar en bruker sette sammen enkle setninger. Den er kompatibel med en øyesporingsenhet som gjør at en bruker kan navigere ved å kun bruke øyene. En som tar i bruke denne interaksjonsformen har tilgang på alle de samme funksjonene som en datamus bruker. Når en bruker har skrevet en setning har mulighet for å presentere dette som lyd. Ettersom programmet er tiltenkt barn, er vært ord representert med et symbol. En utvikler kan enkelt bytte ut eller legge til nye ord og symboler ved å legge de inn i den medfølgende tekstfilen. Programmet er bygget på en moderne teknologi og har en kodekvalitet som gjør at den kan brukes i av andre i fremtidige prosjektet. Det er også mulighet for å videreutvikle prototypen til en ferdig programvare. 

2) I prototypen har vi lagt til flere animasjoner og lyder for å forbedre brukervennligheten. Vi har implementert ulike animasjoner og lagt til forskjellige lyder for differensiere mellom de ulike brikkene. Vi har også gjort det mulig for en bruker å tilpasse ulike aspekter i programmet etter egne ønsker. Dette er også mulig ved bruk av øyesporing. 

For å finne ut hvilken påvirkning dette hadde å brukerene ble det foretatt testing. Vi fikk ikke til å rekruttere folk som bruker Sono Flex eller lignende systemer til dagligdags. Dette gjorde at testen ble foretatt på barn i samme aldersgruppe på målgruppen. Testingen viste indikasjoner på at animasjoner og lyd gir brukeren en bedre forståelse av programvaren. Men får å kunne konkludere må det mer omfattende testing til.


\section{Videre arbeid}

\textbf{Omfattende testing}
I oppgaven har vi hatt vanskeligheter med å rekruttere testere. Først og fremst ble det ikke mulighet for å teste på målgruppen, altså barn i alderen 6 år som dagligdags bruker programvare til kommunikasjon. Og de deltakerne vi fikk, var ikke nok til å konkludere. Det vi derimot fikk, var vage indikasjoner. Så for å få bekrefte eller avkrefte disse, må det mer omfattende testing til. 

\textbf{Legg til talespråk}
Foreløpig er naturlig tale kun tilgjengelig på engelsk. Det er enkelt å legge til flere språk i koden så lenge de er installert på maskinen. 

\textbf{Legg til flere innstillinger som  bruker kan endre}
Det er mulig for en bruker å tilpasse animasjonsfart, symbolstørrelse og temaer. Det burde derimot være mulighet for en bruker å endre mye mer ved kun øyestyring. Som for eksempel mulighet til å designe egne kategorier og flytte symboler. 

\textbf{Hva som skal til for videre bruk} 
Den viktigste funksjonaliteten er tilgjengelig i prototypen og den har blitt testet på barn og studenter. Men det er fortsatt mye som må gjøres for at den kan sies å være ferdig. Blant annet så har Sono Flex mye funksjonalitet som fortsatt ikke er implementert i prototypen. Den har blitt testet, men ikke på brukere som faktisk har behov for den. Så videre arbeid for å ferdigstille prototypen bør fokusere på implementasjon av manglende funksjonalitet og testing på målgruppen.

\textbf{Kategorisering og organisering}
En del av oppgaven var å se på optimal kategorisering og organisering. Dette har det desverre ikke blitt tid til, ettersom dette viste seg å være mer omfattende en først tiltenkt. Men for at et barn effektivt skal kunne bruke Sono Flex eller lignende programvare mer effektivt, bør dette feltet utforskes mer.


Mye av denne oppgaven er basert på videre arbeid, eller legge til rette for videre arbeid. Det kan sies at vi har laget en plattform for videre arbeid som både Tobii og fremtidige studenter kan forske på.


Prototypen inneholder funksjonalitet som gjør at den fungerer, den har blitt testet og både på barn og studentet, med og uten øyesporing. Allikevel vil det alltid dukke opp bugs og fail og v

Brukertilpasning er det mye som mangler

Lag tester til koden. 

Videre testing av de allerede implementere animasjonene

Hva skal til for at det er en ferdig programvare.
    - Legge til flere brikker, dette er enkelt.
    - Sono flex har mange flere funksjoner som ikke er tilgjengelig. Disse må implementeres. 
    - Mye mer testing.
    - 


Som en oppsummering så kan en si at med så lite testpersoner så kan man ikke si noe sikkert, men vi kan konkludere med at prototypen fungerer og at et barn i seks år alderen enkelt greier å uttrykke seg via denne. I tillegg er det indikasjoner på at animasjoner gjør at en ny bruker av programvaren bruker lengre tid enn uten animasjoner. Det er også indikasjoner på animasjoner og lyd gir brukeren en bedre forståelse for hva de ulike typene symbol representerer.  

For å utforske mulige løsninger som kan øke brukervennligheten til Sono Flex  vil en rekke funksjoner og design implementeres og testes. 

Hovedsakelig skal vi se på hvordan ulike elementer kan hjelpe en bruker i å finne ønsket symbol raskere. For eksempel hva vil gjøre at et barn enklere husker hvilken kategori han må trykke på for finne ønsket symbol. For å gjøre dette vil fokuset ligge på audiovisuelle hjelpemidler, men det vil også ses på hvordan en bør organisere de forskjellige ordene nedenfor. Listen nedenfor gir en  beskrivelse av det vi ønsker å utforske.



fungerer sammen med en øyesporingsenhet og mus, slik at en bruker kan velge alle funksjoner som er tilgjengelig for mus er også tilgjengelig ved åp kun bruke øyet.

Programmet er bygget på en moderne teknologi som gjør at den enkelt kan brukes i fremitdige prosjekter blant annet for masterstudenter eller for selskapet. Det er også mulighet for å ferdigstille den til et ferdig produkt.-



Vi har utviklet en prototype som lar barn sette sammen enkle setninger kun ved hjelp av øyene. Prototypen har kommet som et resultat av at den eksisterende løsningen, Sono Flex, er implementert i en teknologi ikke var kvalifisert for det vi ønsket å utforske i andre del. En stor del av oppgaven ble derfor å utvikle denne fra bunn av. Tobii så også muligheten til at dette kunne bli en plattform for fremtidige oppgaver og muligens starten på en ferdig programvare. 
























Den endelige koden er derfor av god kvalitet og det skal være mulig for andre utviklere å forstå og dermed videreutvikle. 


2) I prototypen har vi implementert de fleste komponentene som vi på forhånd ønsket å utforske. 


//Utviklet animasjoner og lyder


//Lar barn bruke øyesporing til å kommunisere, snakker med øyesporingsenheten.
//Koden har ved bruk av mønster, kodekonvesjoner og generelt betenksom rundt koding holdt en god kode.
//Tar i bruk de eksisterende symbolene.
//enkelt å legge til nye symboler.




Tobii Dynavox ønsket teste programvaren på en ny teknologi 

mulige forbedringer på en av deres programvarer. I dette tilfelle Sono Flex. Problemet var at den teknologien som denne er implementer i, holder på å bli utfaset. Derfor ble oppgaven todelt. Eller ret


Oppgaven gikk opprinnelig kun ut på å utforske mulige forbedringer vi kunne gjøre på den programvaren Sono Flex. 


Målet med denne oppgaven var todelt, først måtte vil utvikle en high-fidelity prototype med å


Målet med oppgaven har derfor blitt todelt. Det vil si at i tillegg til å bygge ut funksjoner og teste disse på den eksisterende programvaren, vil også programvaren som disse skal fungere på utvikles. Første delmål av oppgaven vil derfor være å utvikle programvaren. Andre delmål vil være å teste ulike design og funksjoner.


I oppgaven har vi gjort to ting, vi har laget en high-fidelity prototype 































I dette kapittelet vil vi først evaluere de to delmålene som er presentert i det første kapittelet. Her vil vi også komme med forslag til videre arbeid. Til slutt vil det være en konklusjon.

\section{Prototype}

Første del av oppgaven var å utvikle en high-fidelity prototype med fokus på hovedfunksjonaliteten til Sono Flex. Som vil si at den ligner svært mye et ferdig produkt med mye detaljer og funksjonalitet. Dette gjør at en kan foreta konklusjoner om oppførselen til det ferdige produktet. Grunnen til at det måtte utvikles en prototype istedenfor å videreutvikle på den eksisterende var:

- Teknologien som den eksisterende programvare Sono Flex er bygget på, er ikke tilrettelagt for animasjoner. Noe som var en viktig del av oppgaven.

- Den er også blitt satt i vedlikeholds modus av utvikleren Microsoft. Som vil si at feil og bugs vil bli rettet opp, men at det vil ikke komme nye funksjonalitet. Et tegn som kan tyde på at den fases ut.

Prototypen ble utviklet på WPF en teknologi som har eksistert lenge nok til at den er moden samtidig som den fortsatt blir utviklet av Microsoft. Teknologien ble valgt 







//Valget falt på WPF ligger tett opp til den eksisterende teknologien, men er bedre Støtte for hardware

//Kodekvalitet, fulgt kodestandarder, mønster som passer til teknologi. 


a prototype that is quite close to the final product, with lots of detail and functionality. From a user testing point of view, a high-fidelity prototype is close enough to a final product to be able to examine usability questions in detail and make strong conclusions about how behavior will relate to use of the final product.