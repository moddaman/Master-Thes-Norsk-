\chapter{Konklusjon}

\section{Konklusjon}

Oppgaven har bestått av to delmål. Det første var å utvikle en high-fidelity prototype. Det andre var å implementere ulike animasjoner og legge til lyder for å se hvilken innvirkning dette hadde på den tiltenkte målgruppen. Vi skulle også gjøre det mulig for en bruker å  tilpasse programmet ved øyesporing. 

1) Vi har utviklet en high-fidelity prototype som lar en bruker sette sammen enkle setninger. Den er kompatibel med en øyesporingsenhet som gjør at en bruker kan navigere ved kun å bruke øyene. En som tar i bruke denne interaksjonsformen har tilgang på alle de samme funksjonene som en datamus bruker. Når en bruker har skrevet en setning har han mulighet for å presentere dette som lyd. Ettersom programmet er tiltenkt barn, er hvert ord representert med et symbol. En utvikler kan enkelt bytte ut eller legge til nye ord og symboler ved å legge de inn i den medfølgende tekstfilen. Programmet er bygget på en moderne teknologi og har en kodekvalitet som gjør at den kan brukes av andre i fremtidige prosjektet. Det er også mulighet for å videreutvikle prototypen til en ferdig programvare. 

2) I prototypen har vi lagt til flere animasjoner og lyder for å forbedre brukervennligheten. Vi har implementert ulike animasjoner og lagt til forskjellige lyder for differensiere mellom de ulike brikkene. Vi har også gjort det mulig for en bruker å tilpasse ulike aspekter i programmet etter egne ønsker. Dette er også mulig ved bruk av øyesporing. 


For å finne ut hvilken påvirkning dette hadde å brukerene ble det foretatt testing. Det ble først forsøkt å rekruttere barn som brukere Sono Flex eller lignende systemer til daglig, men til tross for gode kontakter og flere forespørsler fikk vi ingen respons. Det ble rekruttert barn i samme alder, men som ikke hadde noe erfaring med øyesporing. Testingen viste indikasjoner på at animasjoner og lyd gir brukeren en bedre forståelse av programvaren. Men det var altfor få testere til å kunne konkludere. For å kunne gjøre det må det mye mer omfattende testing til. Det som testingen bekreftet var at prototypen fungerte svært bra. Barna fikk til å skrive enkle setninger med øyesporing og mus.


\section{Videre arbeid}

\textbf{Omfattende testing}
I oppgaven har vi hatt vanskeligheter med å rekruttere testere. Først og fremst ble det ikke mulighet for å teste på målgruppen, altså barn i alderen 6 år som til daglig bruker programvare til kommunikasjon. Og de deltakerne vi fikk, var ikke nok til å konkludere. Det vi derimot fikk, var vage indikasjoner. Så for å få bekrefte eller avkrefte disse, må det mer omfattende testing til. 


\textbf{Internasjonalisering}

\textbf{Legg til talespråk}
Foreløpig er naturlig tale kun tilgjengelig på engelsk. Det er enkelt å legge til flere språk i koden så lenge de er installert på maskinen. 

\textbf{Legg til flere innstillinger som  bruker kan endre}
Det er mulig for en bruker å tilpasse animasjonsfart, symbolstørrelse og temaer. Det burde derimot være mulighet for en bruker å endre mye mer ved kun øyestyring. Som for eksempel mulighet til å designe egne kategorier og flytte symboler. 

\textbf{Hva som skal til for videre bruk} 
Den viktigste funksjonaliteten er tilgjengelig i prototypen og den har blitt testet på barn og studenter. Men det er fortsatt mye som må gjøres for at den kan sies å være ferdig. Blant annet så har Sono Flex mye funksjonalitet som fortsatt ikke er implementert i prototypen. Den har blitt testet, men ikke på brukere som faktisk har behov for den. Så videre arbeid for å ferdigstille prototypen bør fokusere på implementasjon av manglende funksjonalitet og testing på målgruppen.

\textbf{Kategorisering og organisering}
En del av oppgaven var å se på optimal kategorisering og organisering. Dette har det desverre ikke blitt tid til, ettersom dette viste seg å være mer omfattende en først tiltenkt. Men for at et barn effektivt skal kunne bruke Sono Flex eller lignende programvare mer effektivt, bør dette feltet utforskes mer.



































































