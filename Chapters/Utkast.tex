
\section{UTKAST: ikke les}


Som med animasjon kan lyd både virke for og i mot sin hensikt. Horanger og Nielsen fant ut under testing av nettsider at ungdommer og voksne fant kvitre og kime lyder var irriterende, men at de appellerte til barn. 

Sammen med animasjoner kan lyder i programvare være effektive til å gi brukeren tilbakemelding på en handling. Eksempelvis kan lyder brukes til varsle brukeren om at de har gjort et feil valg \cite{}, men som med animasjoner vil for mange lyder og uventede lyder være distraherende og irriterende. Nielsen og Loranger fant 




I prototypen ønsker animasjonen å vise hva som skjer,  ved at symbolet beveger seg fra opprinnelige posisjon til set

I Sono Flex dukker symbolet bare opp i setningslisten.

I symboltabellen er det 3 forskjellige typer. Man har 
ordsymbol, kategorisymbol og navigasjonssymbol. Et ordsymbol representerer selvsagt et ord og vil legge seg i listen sammen med andre ord som vil bli 

For å vise hva som skjer, fremfor at det bare skjer. I den eksisterende løsningen Sono Flex var det ingen for animasjoner. Slik at når brukeren trykte på en knapp, skjedde responsen umiddelbart. Man kan si at animasjonene skal representere det som skjer med dataen. Eksmepelvis hvis brukeren trykker på en knapp 


//tre ulike knapper
    //Trykk på knapp
    //neste side
    //Organisasjon



Det vil derfor være viktig at animasjonene som er implementert gir verdi til applikasjonen fremfor å distrahere.

Det vil derfor være viktig å finne ut om animasjonene gir noe verdi 

Interaksjonsdesigneren Nielsen har gjort flere undersøkelser på hvordan en skal få optimal brukervennlighet på webapplikasjoner. 

Animasjoner i programvare og nettsider blir ofte snakket om i negative ordlag fordi de

for det de ofte blir brukt overdrevet mye og er u

Animasjoner blir ofte sett på som en distraksjon når det blir brukt i applikasjoner for det de fjerner fokuset fra det som er viktig. Som 


For å kunne sjekke hvordan animasjon påvirket et barns evne til å samhandle med applikasjonen, måtte de ble implementert.  

//visualiserer at symbolet går fra griden til output, hjelper barnet
// Vanskeligheten med å skaffe empirisk data


Dataanimasjon, animasjonsfilmteknikk som gjør bruk av elektronisk databehandling for å skape illusjon av bevegelse, har fått en svært sentral plass innenfor reklamefilm og fjernsyn. Her har utviklingen gått fra enkle todimensjonale tegninger til avansert 3-D-animasjon som ikke kan fremstilles på andre måter enn med datamaskin.

Ifølge Store norsk leksikon defineres animasjon som levendegjørelse, besjeling, oppmuntring, tilskyndelse. 


Applikasjonen tilbyr flere animasjoner som brukeren kan velge mellom, enten alene eller flere i kombinasjon. Noen animasjoner kan ikke være operative samtidig ettersom det vil føre til konflikt. 

\subsection{Fade in, fade out}

Gjør rede for og forklar de ulike animasjonen

\section{Navigasjon}
Gjør rede for og forklar de forkjellige navigasjons hjelpemidlene

\section{Personlig tilpasning}
Forklar hvordan applikasjonen legger til rette for brukeren.
    - Temaer, forkjellige farger, lyder, 
    - Bruker kan velge hvilke animasjoner som ønskes og hurtigheten på dem.
    
\section{Lyd}
    Gjør rede for de ulike lydene.

\chapter{Dette er bare utkast og stikkord}

Rapporten vil ikke gi noen videre beskrivelse av C-Sharp, derimot vil WPF bli forklart.


\section{Teknologier}

Øyesporingsenheten beskrevet i kapittel  krever at systemet kjører på operativ systemet Windows og APIet som det leveres enheten leveres med,  fungerer kun i programmeringspråkene C-Sharp og C++. Med dette som utgangspunkt ble resten av teknologivalgene er basert på anbefalinger fra dokumentasjonen og tidligere erfaring. 


\begin{description}
  \item[IDE] Visual Studio
  \item[Rammeverk] .NET
  \begin{description}
     \item[Programmeringspråk] C\#
     \item[Grafikk] Windows Presentation Foundation 
\end{description}
  \item[Versjonskontroll] Git (BitBucket)
\end{description}


Fitzergald key, personliggjør keyboardet med farger. Viktigste er at det er konsistent


For å kunne bruke symboler som en kommunikasjonsform kreves det at brukeren har mulighet til samhandle med dem. For å få til dette brukes det en øyestyringsenhet koblet til en datamaskin som fanger opp brukerens pupill bevegelser. På den måten erstatter øyene den vanlige musepekeren.

\subsubsection{Kalibrering}

Første gang en kobler øyestyringsenheten til maskinen, anbefales det at brukeren gjør en oppmåling og kalibrering for mer presis øyesporing. Oppmålingen innebærer at personen måler størrelsen på skjermen - kalibreringen at det brukeren bes om å følge en prikk som traveserer skjermen. Ettersom dette vil gi øyestyringsenheten et referansepunkt. Prosessen er kun nødvendig første gang for gjeldene bruker. 

Dette gir tilgang på deres application programming interface (API), herved referert til som Tec API. Tec API gjør funksjoner og informasjon fra sporingsenheten tilgjengelig. 


The Tobii Eye Control API provides two alternative access points: a .NET assembly and a C dynamic-link library. Both alternatives
give the developers of eye controlled applications access to the functionality of the Gaze Interaction Server. The .NET API also
contains specialized interaction support for common user interface frameworks such asWindows Presentation Foundation
(WPF) andWindows Forms. The C interface, called MPACI, provides backward compatibility with older Tobii products such as
the MyTobii software.


\subsection{Systemkrav}

\subsection{Nøkkelfunksjoner og konsepter}

\section{}



Første gang en kobler til må en gå igjennom ett oppsett. Dette innebærer måling av den aktuelle skjermen som brukes og en kalibrerings rutine. For å oppnå nøyaktig øyesporing.  

For en førstegangsbruker av enheten anbefales det å gjøre en kalibrering, for optimal opplevelse. Informasjonen


Øyestyringsenheten som er illustrert i figur \ref{fig:tobiiPc}  . Denne



\subsection{Utfordringer}

En utfordring ved øyesporing er blunking. Blunking er et er en kortvarig sammentrekning av øyelokket, noe som gjør at øyet ikke vil gi refleksjon som kamera kan fange, og derfor ikke ha koordinater på hvor brukeren ser. Dette løses under analysen. Ved at og bruke koordinatene og øyets hurtighet før øyelokkene trakk seg sammen kan man ekstrapolere seg fram til en tilnærmet korrekt fiksering. 


\subsection{Tobii Software Developer Kit}
\label{subsec:blikk}

Sammen med Tobii PCeye go følger det med programvare for å kontrollere applikasjoner som bruker sporingsenheten. De ulike komponentene er beskrevet nedenfor. I denne rapporten vil kun førstnevnte være relevant. Ettersom applikasjonen er spesialisert til blikkinteraksjon og ikke er en standard Windows applikasjon.

\subsubsection{Blikk interaksjonsserver}
en sentral HUB som tilbyr klient applikasjoner øyesporingsdata. 

\subsubsection{Blikk interaksjons innstillinger}
Et kontrollpanel for interaksjons innstillinger og oppgaver relatert til øyesporing.

\subsubsection{Windows Kontroll}
Tilbyr blikk interaksjon for standard Windows applikasjoner, ergo vanlige applikasjoner som ikke er laget for øyesporings interaksjon.


\subsection{Blikk Interaksjonsserver}

Blikk interaksjons-serveren er som nevnt i seksjon \ref{subsec:blikk} en HUB for programvaren, eller kjernen. Hovedformålet til denne komponenten er å samhandle med Øyesporingsenheten og tilby interaksjonsfunksjonalitet for å kontrollere Windows baserte applikasjoner som vil bruke sporingsdata. I tilegg er det denne komponenten som tar seg av kalibrering, sporingsstatus - samt bruker og applikasjons innstillinger.


\section{eXtensible Application Markup Language} 
 
 WPF bruker eXtensible Application Markup Language (\gls{XAML}), som er et XML-basert språk til å definere og kombinere ulike grensesnittelementer. Figur \ref{lst:myLabel} viser hvordan et vindu med en knapp er definert i XAML. Resultatet kan ses i figur \ref{fig:xamlButton}.  
 
 
For å kunne samhandle med de grafiske objektene definert i XAML filen opprettes det alltid en tilhørende kodefil sammen med denne. Kodefilens formål er å ta seg av data og logikk for å ha et klart skille mellom utseende-spesifikk kode og oppførsel-spesifikk kode. For eksempel så vil en knapp sitt utseende og plassering bli definert XAML filen, mens hvordan data blir påvirket av interaksjon med knappen blir definert i den tilhørende kodefilen. 
 
WPF oppfordrer til å skille mellom utseende spesifikk kode og oppførsel-spesifikk kode ved at grafiske elementer skrives i XAML og interaksjonen  kun brukes til generere  
XAML brukes til å generere brukergrensesnitt,  
,Hver XAML-fil har en tilhørende kode-fil for håndtering av hendelser og oppførsel. Dette gjør at koden har et klart skille mellom utseende-spesifikk kode og oppførsel-spesifikk kode. Den tilhørende kode-filen til XAML koden beskrevet i kodesnutt \ref{lst:myLabel} er vist i kodesnutt \ref{lst:backbutton}. Ved å trykke på knappen bildet vil metoden button\textunderscore Click i kodefilen bli kalt og en dialogboks vist til brukeren. 
 