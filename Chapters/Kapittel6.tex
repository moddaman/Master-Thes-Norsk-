
\section{Rekruttering av testere}

For å rekruttere deltakere til testen ble først pedagogisk avdeling ved høgskolen i Bergen
kontaktet. De hadde tidligere hatt erfaring med å teste på den ønskede målgruppen, og målet var
at disse kunne sette oss i kontakt med nøkkelpersoner og i tillegg gi retningslinjer og tips
angående det å bruke barn med funksjonshemninger til undersøkelser. Gjennom dette møte ble vi
satt i kontakt med Pedagogisk-psykologisk Tjeneste(PPT). PPT er kommunal tjeneste som skal
hjelpe barn, ungdom og voksne som strever i utviklingen, eller som har en vanskelig
opplæringssituasjon. Blant annet ved å gi systemrettet støtte, med råd og veiledning til skoler om
pedagogisk ledelse av gruppe- og læringsmiljø, og bistand med kompetanse- og
organisasjonsutvikling. De hadde god erfaring med målgruppen og hadde et større kontaktnettverk
enn ved høgskolen.
Gjennom PPT ble det sendt ut et informasjonsskriv til foresatte av barn som passet inn i
målgruppen hvor det ble gjort rede for hvordan undersøkelsen ville bli gjennomført og hvilke og
hvor lenge data ville bli lagret. For at testingen skulle gjennomføres måtte en av barnets foresatte
signere og returnere skjemaet. Uheldigvis var det ingen av de foresatte som ønsket at barnet
skulle delta i undersøkelsen.
Gruppen ble derfor enige om at programvaren uansett måtte testes og at det nærmeste da ville
være barn i samme alder som målgruppen, men uten funksjonshemningene.
Det ble etter en kontaktrunde klart at 6 foresatte med barn ved Krohnengen barneskole hadde gitt
sin tillatelse til å la barna delta i undersøkelsen.

\section{Gjennomføring av test}

\subsection{Pilottest}

Før testingen ble det utført en pilottest på skolen med medstudenter. Dette ble gjort for at testingen
skulle gå best mulig og for å identifisere tekniske feil og eventuelle andre hindringer som kunne
oppstå.

\subsection{Sted}
Eksperimentet ble utført på Krohengen barneskole inne på biblioteket mellom klokken 13:00 -
16:00. Dette rommet ble valgt fordi det var nært lokale hvor barna oppholdt seg og fordi det var
stille. Det var ingen andre barn utenom deltakeren i rommet når testen ble utført.
(Bilde av testrom og utstyr) 

\subsection{Testgjennomføring}
Hver test ble gjennomført ved at et og et barn kom inn i biblioteket hvor det først ble foretatt en kort
samtale for at barnet skulle føle seg komfortabel. Under testingen var kun deltaker og testleder
tilstedeværende for unngå distraksjoner. Barnet ble så satt foran datamaskinen og øyesporingsenheten satt slik at den siktet på deltakerensansikt. For at øyesporingsenheten skulle bli mest nøyaktig måtte den tilpasses for hver bruker ved
en kalibreringsprosess. Prosessen foregikk ved at brukeren følger en rød prikk som starter i
venstre hjørne og traverserer over skjermen to ganger før den stopper nede i høyre hjørne. Dette
tar cirka 30 sekund. Etter hver test ble det gitt en farge som beskskrev hvor godt enheten greide å
tilpasse seg brukeren. Rødt er dårligst og programvaren vil ha vanskeligheter med å regne ut hvor
brukeren ser på skjermen. Gul vil si at kalibreringen gikk greit og at brukeren skal kunne bruke
enheten på en god måte. Grønn er den mest nøyaktige kalibreringsgraden og enheten vil presist
fange opp hvor brukeren skuer. Hvis en bruker fikk kalibreringsgrad rød, ble prosessen gjentatt en
gang. Hvis det igjen viste seg å bli rød ble brukeren bedt om å bruke mus. Kalibreringsgraden ble
notert for alle brukerne. Deretter ble programvaren startet, og brukeren fikk lov å utprøve programvaren før selve testen startet.

Testen startet med at brukeren ble  gitt 5 oppgaver som han skulle gjennomføre. Hver oppgave hadde en maks
tid på 5 minutt. Det vil si at hvis deltakeren brukte mer enn dette ville oppgaven bli registrert som feil og avsluttet. Brukeren ville så bli tildelt neste oppgave. Denne makstiden valgt utifra pilottestingen, der brukte deltakerne ca 30 sekund per oppgave, med tanke på at disse var i 20 årene ble det også lagt til en god sikkerhetsmargin. Under testingen ble det forsøkt å gi minst mulig spørsmål og ledetråder for å ikke forstyrre eller hjelpe brukeren, ettersom dette ville ha påvirket resultatet. Når brukeren hadde gjennomført alle oppgavene ble han stilt 5 spørsmål. Deretter ble deltakeren
takket for deltakelsen og tildelt et trekk til sykkelsete som gave.

\section{Datainnsamling}

For å samle inn data ble det brukt flere ulike metoder. 


\subsection{Loggfører}
En viktig del av undersøkelsen var å se hvor lang tid og hvor mange trykk deltakeren brukte på å utføre de forskjellige oppgavene som han ble tildelt. For å gjennomføre dette ble det brukt en logger som lagret data for hver test på separate tekstfiler. Grunnen til at en logger ble brukt er fordi data lagres i et standardformat som gjør at en enkelt kan kjøre spørringer mot det og dermed hente ut målbar data. Data lagres også med en presisjon på tid som en menneskelig observatør ikke har mulighet til og er ikke like mottakelig for feil. Den har også den fordelen at den ligger i bakgrunnen uten brukeren merker noe til den samtidig som den ikke har noen merkbar innflytelse på programvarens ytelse.

Så for hver test blir det først logget starttidspunktet, om animasjoner er på eller av, alderen til barnet og om den nødvendige kalibreringen er blitt fullført. Deretter er det kun data fra interaksjoner brukeren gjør med programvaren som blir lagret. Ettersom brukeren kun har mulighet til å trykke på brikker kan det forenkles til at hver gang deltakeren trykket på en brikke ble data logget. Så for vært for vært trykk på en brikke ble det lagret hva som sto på brikken, hvilken type brikken var av og tidspunktet.


\subsection{Skjermopptak}

Loggeren fanger kun opp interaksjoner deltakeren gjør med programvaren, noe som utelater det som skjer imellom hver interaksjon. For å kompensere for dette ble det bestemt å ta i bruk en skjermopptaker som kunne fange opp hendelser mellom trykk. Noe som kunne være interessant for å se hvor brukeren nesten trykket og hvordan han beveget blikket, og generelt andre unntak som måtte oppstå. En stor ulempe med skjermopptak er at en observatør fysisk må se igjennom opptaket for å kunne hente ut informasjon, imotsetning til loggeren, hvor store mengder data enkelt kan hentes ut og måles opp mot hverandre.

Skjermopptakeren ble som med loggeren startet på nytt for hver test og lagret som seperate mp4 filer. E


\subsection{Observasjon}

Som en del av det å senke terskelen for at foresatte skulle tillate barna å delta i eksperimentet ble det valgt å ikke filme barna under testing. Dette gjorde derimot at en ikke fikk med seg reaksjoner eller spørsmål som ble stilt underveis. Så for å kunne fange opp dette ble det brukt en observatør til å notere ned eventuelle hendelser og spørsmål som deltakeren måtte gjøre seg underveis. Ulempen med dette er at man ikke har samme mulighet som med film til å undersøke det i ettertid. Hvis man går glipp av noe er det ikke mulig å spole tilbake for å se det igjen. Det kan også virke forstyrrende på en deltaker at en person noterer det han foretar seg. 


\subsection{Intervju}


Intervju
Intervjuet ble hovedsakelig basert på spørsmål laget på forhånd, men for å kunne ofte var svarene
svært vage og det ble derfor nødvendig å spørre mer utdypende spørsmål.
Oppgaver
Under eksperimentet ble deltakerne gitt flere oppgaver som de skulle gjennomføre. Disse var de
samme for alle deltakerne og gitt i samme rekkefølge for at forutsetningene skulle være de
samme.
De første oppgavene gikk i at brukeren kun skulle finne et gitt ord og var som følger:
1. Finn ordet “hvordan”
2. Finn ordet “banan”
3. Finn ordet “Elg”
4. Finn ordet “april”
Deretter ble deltakerne spurt om de kunne skrive setninger. Der noen inneholder ord fra de første
oppgavene.
1. Jeg vill ha banan
2. Jeg liker katt
3. Jeg kan hoppe
4. Det er hennes katt
Resultater
Oppgaver
(tabell 1)
Intervju
http://www.udir.no/Regelverk/tidlig-innsats/Skole/Oversikt-over-aktorene/PP-tjenesten/ 