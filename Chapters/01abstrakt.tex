
Denne rapporten tar for seg prosessen rundt utviklingen av en high-fidelity prototype basert på en eksisterende programvare og testing av nye funksjoner for økt brukervennlighet. 

Tobii Dynavox har en programvare som heter Sono Flex som skal hjelpe unge mennesker som helt eller delvis mangler tale å kommunisere ved hjelp av øyesporing. De ønsket å utforske to ting.  Muligheten for å utvikle programvaren på en mer moderne plattform. Implementere og teste hvilke påvirkning animasjoner og lyd har på målgruppen.

Programvaren ble bygget på en ny plattform med hovedfunksjonaliteten tilgjengelig. Koden har følgt en såpass god standard at det skal være mulig å videreutvikle den til et en fullstendig programvare. I tilegg til hovedfunksjonaliteten ble det også implementert flere animasjoner. 

Til slutt ble det også utført en test på programvaren. Denne ga ikke nok data til å konkludere hvorvidt animasjoner og lyd ga økt brukevennlighet, men nok til at en person kan bruke den til kommunikasjon.



1