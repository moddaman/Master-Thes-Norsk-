\chapter{Animasjoner, lyd og brukertilpasning}
 
\section{Animasjon} 
 
 
Animasjoner i programvare brukes som et virkemiddel for å tiltrekke seg oppmerksomhet, underholde eller demonstrere noe. Det kan enten være en kort animasjon der en knapp tiltrer til en side når man svever over den med musen eller det kan være av den lengre sorten som en flash-video.  Animasjon er i hovedsak alt som beveger seg i applikasjonen. Selv om dynamikken som animasjoner kan fungere som et positivt virkemiddel viser undersøkelser gjort av Nielsen og Loranger \cite{NielsenBok}  at for mye blinkende og bevegende elementer kan slite ut brukeren og gjør det vanskeligere å fokusere på oppgaven. Med animasjonene som er implementer ønsker vi å undersøke dette. Hvilke animasjoner fungerer som et hjelpemiddel og med det gir verdi til applikasjonen og hvilke som er distraherende for brukeren og gir en uønsket effekt. 
 
 
I denne prototypen skjer animasjonene som en respons på at brukeren trykker på et symbol. Hva som skjer avhenger av hvilken type symbol det er. De ulike typene som finnes i symboltabellen ble forklart i seksjon \ref{subsubsec:symboltabell}, og er ordsymbol, kategorisymbol og navigasjonssymbol. 
 
 
 
 
\subsection{Animasjoner: ordsymbol} 
 
 
Når en bruker trykker på et ordsymbol, vil symbolet legge seg i setningslisten og er da klar for gjøres om til tydelig tale. Når brukeren har fullført setningen i Sono Flex blir brukeren presentert for endringen ved at symbolet vises i setningsfeltet. Med prototypen ønsker vi  å gi en mer visuell representasjon av denne endringen til brukeren. Ved å la brukeren kunne velge mellom to animasjoner som oppstår når han trykker på ønsket ordsymbol. De to animasjonene fungerer som følgende: 
 

I) Brukeren trykker på et ordsymbol. 
   Ordsymbolet krympes til 5 prosent av størrelsen. 
   I setningslisten vil en kopi av det krympede ordsymbolet dukke opp. 
   Kopien forstørres så til den opprinnelige størrelsen. 
   Ordsymbolet som opprinnelige ble trykket blir forstørret til opprinnelig størrelse. 
 

II) Brukeren trykker på et ordsymbol. 
    Ordsymbolet beveger seg ut av sin posisjon og glir mot den posisjonen den vil ha i setningslisten. 
    Når den har truffet posisjonen i setninglisten, stopper den opp og blir værende i ro. 
 
 
    Med disse animasjonen vil en først og fremst gi brukeren beskjed om at et valg har blitt gjort, men de ønsker også gjøre bruk av programvaren mer spennende.  I Sono Flex  vil ingen av knappene ha noe visuelt som gjør at brukeren kan skille hva som skjer, utenom at symboltabellen bytter ut de eksisterende symbolene. I dette tilfelle vil en se at symbolet går fra å være en statisk symbol til å bli flyttet til setningslisten og bli en del av ønsket setning.  
 
 
 
 
 
 
\subsection{Animasjon: Kategorisymbol} 
 
 
Når en bruker trykker på et kategorisymbol vil symbolene i symboltabellen byttes ut med symbolene som hører til i den gjeldene kategorien. Hvis en trykker på kategorien "dyr" så vil tabellen fylles med ordsymbol som "Hund", "katt", "tiger" o.s.v. Her ønsker animasjonen å hjelpe til med å fortelle at man ikke trykker på ordet dyr, men kategorien dyr og at brukeren får en forståelse for forskjellen mellom dem. Animasjonen startet ved at symbolet som brukeren trykket på forstørres helt til den dekker hele symboltabellen. Det vil si at ingen av de andre symbolene vises, kun kategorisymbolet. Symbolet vil så igjen forminskes, men symbolene i tabellen vil være erstattet av symbolene som tilhører kategorien.  
 
 
 
 
\subsection{Animasjon: Navigasjonssymbol} 
 
 
Den siste typen symbol er navigasjon og ved interaksjon vil den navigere mellom sider når det ikke er plass til alle symbolene på en side. Det vil si at symbolet skal kunne navigere fremover, fra side 1 til 2 og 2 til 3 og bakover igjen. Når brukeren trykker fremover starter animasjonen ved at alle symbolene som er på gjeldene side og neste side begynner å bevege seg mot venstre. Slik at symbolene på gjeldene side kontinuerlig blir byttet ut med de på neste. Symbolene på siden sklir ut, mens symbolene på neste sklir inn og overtar plassene. Når brukeren nå trykker på bakover-symbolet vil det samme skje bare at symbolene sklir mot høyre. 
 
 
\section{Lydeffekter} 
 
 
Som en del av rapporten ønsket vi å undersøke hvilken påvirkning lydeffekter hadde på brukeren. Hovedsakelig var målet å finne ut to ting. I hvilken grad hjelper effektene brukeren i å navigere rundt i applikasjonen og om det vil påvirke brukerens helhets uttrykk av applikasjonen.  
 
 
\subsection{Lydeffekter i prototypen} 
 
 
I prototypen er det lagt til flere lydeffekter, utelukkende i form av små lydklipp på maks 1 sekund. Effektene vil kun bli avspilt som en respons til noe brukeren foretar seg eller som å informere om en hendelse. Denne begrensning finnes for å ikke skape forvirring hos brukeren. Hvis lyd også hadde blitt avspilt tilfeldig så ville meningen med effektene forsvinne. Fordi brukeren blir lurt til å tro at programvaren har registrert en interaksjon, selv om han ikke har foretatt seg noe. Det er derfor viktig at det er lett for brukeren å skille mellom effektene som representerer en interaksjon og informasjon. De ulike interaksjonslydeffektene er også forskjellige, og hvilken som blir spilt er avhengig av,  som med animasjon,  hvilken type komponent brukeren samhandler med. Eksempelvis vil lyden som blir avspilt når brukeren trykker å tilbaketasten representere noe som forsvinner.  
 
 
 
 
\begin{itemize} 
\item Symbolord - Enkel klikkelyd 
\item Kategoriord - usikker 
\item navigasjonsord - svisj 
\item Innstillinger - Verktøyskasse 
\item tilbaketasten/slett - forvsinner 
\item hjem - usikker 
\item prat 
\end{itemize} 
 
 
 
Dette vil forhåpentligvis gjøre at brukeren mer effektivt kan bekrefte valget og fortsette med oppgaven. Som  
 
Hvilken brukeren blir presentert for er avhengig av hvilket komponent han interagerer med.  Grunnen er at hvis effektene hadde blit avspilt uten at brukeren foretar seg noe, så kan 
 
Hvilken lyd som blir avspilt er avhengig av hvilket valg brukeren har foretatt seg, og responsen til brukeren vil  
 
Eksempelvis hvis brukeren foretar seg et ulovlig valg vil han bli presentert for en lyd som har gir negative assosiasjoner.  D 
 
Lyden som vil bli gitt til brukeren vil komme i form av små lydklipp som maks vil vare i et sekund,  ettersom det skal være nok for at brukeren kan registrere det.  
 
 Tilbakemelding i form av lyd gir brukeren en bekreftelse på at han valget han nettopp har foretatt har blitt registrert og  
 
Ta for eksempel brødristeren. Når den er ferdig vil brødet hoppe opp, men i tilegg vil den gi fra seg et pling. Altså brukeren vil få tilbakemelding på at han kan hente brødet både visuelt(brødet som hopper opp) og gjennom lyd(plinget). Ved å g 
 
Lyd brukes ofte for å fortelle brukeren av produktet noe. For eksempel vil brødristen gi fra seg et pling når den er ferdig samtidig med at skivene hopper opp av maskinen. I dette tilfellet vil  
 
 
I dag brukes lyd ofte i flere sammenhenger, i alt fra 
 
 
 
